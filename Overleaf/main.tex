\documentclass[12pt,oneside,leqno]{article}
\usepackage{amssymb,amsmath,amsfonts}
\usepackage{dsfont}
\usepackage{rotating,bigstrut}
\usepackage{verbatim}
\usepackage{longtable}
\usepackage{enumitem}
% \usepackage{caption}
\usepackage{color,soul}
\usepackage[tiny,center]{titlesec}
\usepackage{footnote}
\usepackage{textcomp}
\usepackage{tabularx, booktabs}
\usepackage{bbm}
\usepackage{lscape}
\usepackage{pdflscape}
\usepackage{amsmath}
% \usepackage{caption}
\usepackage{float}
\usepackage{afterpage}
\usepackage{graphicx}
%\usepackage{subcaption}
\usepackage[table]{xcolor}
\usepackage[merge, elide]{natbib}
\usepackage[flushleft]{threeparttable}
\usepackage{multirow}
\usepackage{textcomp}
\usepackage{subfigure}
\usepackage{listings}
\lstset{
    showspaces=false,         
    showstringspaces=false,
    basicstyle=\ttfamily\footnotesize,
    breaklines=true,
    postbreak=\mbox{\textcolor{gray}{$\hookrightarrow$}\space},
}
 % this one is nice, but not crucial
%\usepackage[font={footnotesize},labelfont={bf},justification=justified,singlelinecheck=false,skip=0pt]{caption}
\usepackage{caption}
% \usepackage[labelfont={bf},singlelinecheck=false,justification=centering,,skip=6]{caption}


\DeclareCaptionLabelSeparator{none}{ }
\captionsetup{labelsep=none}
\captionsetup[figure]{labelsep=period,labelfont={bf},singlelinecheck=false,justification=centering,skip=6pt}
\captionsetup[table]{labelsep=none,labelfont={bf},singlelinecheck=false,justification=centering,skip=6pt}

\linespread{1.5}


\newcolumntype{Y}{>{\centering\arraybackslash}X}
\makesavenoteenv{tabular}
%\makesavenoteenv{table}

\titlelabel{\thetitle.\quad}
\titleformat*{\section}{\center\scshape}
\titleformat*{\subsection}{\center\itshape }
\titleformat*{\subsubsection}{\itshape }
\usepackage{setspace}
\onehalfspacing
\usepackage{float}
\usepackage{afterpage}
\usepackage{graphicx}
% \usepackage{subcaption}

\usepackage{mwe}

\usepackage{tikz}

% \setlength{\paperwidth}{8.5in} \setlength{\paperheight}{11in} \setlength{\textwidth}{6.5in} \setlength{\textheight}{9in}
\setlength{\oddsidemargin}{0in} \setlength{\evensidemargin}{0in} \setlength{\topmargin}{0in}

\setlength{\headheight}{0in} \setlength{\headsep}{0in}
\usepackage[a4paper, top=1.5in, bottom=1.5in, right=1in, left=1in]{geometry}

\newtheorem{assume}{Assumption}
\newtheorem{prop}{Proposition}

% defining column types with fixed widths
\newcolumntype{P}[1]{>{\centering\arraybackslash}p{#1}}

\newcolumntype{L}[1]{>{\raggedright\arraybackslash}p{#1}}

\newcommand\fnote[1]{\captionsetup{font=small}\caption*{#1}}
\renewcommand{\arraystretch}{1.25}
\DeclareMathOperator*{\E}{\mathbb{E}}
\newcommand{\Lagr}{\mathcal{L}}
\usepackage{titlesec}
\newcommand{\note}[1]{}
\makeatletter
\makeatother
\titlespacing{\subsubsection}{0pt}{\parskip}{-\parskip}


\usepackage[titletoc]{appendix}
\usepackage{pdfpages}

\let\oldFootnote\footnote
\newcommand\nextToken\relax

\renewcommand\footnote[1]{%
\oldFootnote{#1}\futurelet\nextToken\isFootnote}

\newcommand\isFootnote{%
\ifx\footnote\nextToken\textsuperscript{,}\fi}

\usepackage[colorlinks=true, allcolors=blue]{hyperref}
\newcommand{\tabfont}{\fontsize{9}{10}\selectfont}

\usepackage{comment}  

\makeatletter
\def\bibsection{\begin{center}\textsc{references}\end{center}}
\renewcommand\@biblabel[1]{}
\makeatother
\setlength{\bibsep}{0pt}        
\setlength{\bibhang}{1.5em}     
\renewcommand{\bibfont}{\small} 

\renewcommand{\tablename}{TABLE}
\renewcommand{\thetable}{\Roman{table}}
\captionsetup[table]{labelfont=sc,textfont=sc,labelsep=newline,
  justification=centering,singlelinecheck=false,skip=6pt}

\renewcommand{\figurename}{FIGURE}
\renewcommand{\thefigure}{\Roman{figure}}
\captionsetup[figure]{labelfont=sc,textfont=sc,labelsep=newline,
  justification=centering,singlelinecheck=false,skip=6pt}
\graphicspath{{Figures/}}


% ============================================================
\begin{document}

\title{\Large \bf Managerial Theories of Technology\thanks{Chai and Van Lent gratefully acknowledge funding from the Deutsche Forschungsgemeinschaft Project ID 403041268 - TRR 266.}\bigskip}
\author{
{\Large Xingxu Chai}\thanks{\textbf{Frankfurt School of Finance and Management}; Postal Address: Adickesallee 32-34, 60322 Frankfurt am Main, Germany; E-mail: x.chai@fs.de}\\
{\Large Ruishen Zhang}\thanks{\textbf{The University of Hong Kong}; Postal address: Pok Fu Lam, Hong Kong SAR; E-mail: ruishen@hku.hk}
\and {\Large Laurence van Lent}\thanks{\textbf{Frankfurt School of Finance and Management}; Postal Address: Adickesallee 32-34, 60322 Frankfurt am Main, Germany; E-mail: l.vanlent@fs.de}\\
{\Large Menghan Zhu}\thanks{\textbf{%
Vrije Universiteit Amsterdam}; Postal Address: De Boelelaan 1105, 1081 HV Amsterdam, Netherlands; E-mail: m.zhu@vu.nl}
}
\date{{February 2026}}

\maketitle
 %\newpage
\thispagestyle{empty}
\vspace{-1cm}
{\setstretch{1.0} }
\begin{spacing}{1.0}
\begin{abstract}
\noindent We extract the causal theories managers hold about new technologies from 450,000 quarterly earnings conference calls (2002--2024) covering 29 disruptive technologies across nearly 20,000 firms in 94 countries.  Using large language models, we identify explicit cause-and-effect claims and organize them into a two-layer taxonomy that distinguishes \emph{why} firms say they adopt a technology from \emph{what} they expect it to deliver.  We document four main findings. First, managers hold heterogeneous causal theories about technology: after absorbing technology-by-time and industry-by-time fixed effects, much of the variation in cause and effect composition is at the firm level. Second, these theories are persistent and portable: firms carry their causal frame from one technology to another.  Third, conditional on the same technology, industry, and time, firms whose managers hold growth-oriented theories invest more, while those whose managers hold efficiency-oriented theories restructure.  Fourth, when a manager's causal frame deviates from the frame that ex post characterizes the technology's actual trajectory, subsequent firm performance tends to be worse. The results suggest that managerial beliefs about how technology works are a source of heterogeneity in firms' responses to technological change.

\smallskip\noindent
\textbf{Keywords:} managerial beliefs; technology adoption; diagnostic expectations; earnings calls; textual analysis; misallocation

\smallskip\noindent
\textbf{JEL codes:} D83, D84, G14, G31, O33
\end{abstract}
\end{spacing}
\thispagestyle{empty}
\clearpage
\setcounter{page}{1}
\doublespacing
% ============================================================
\section*{I. Introduction}

How a firm responds to a new technology depends on what its managers believe about that technology.  A manager who thinks artificial intelligence will open new markets invests differently from one who thinks it will reduce headcount.  Both may be right, both may be wrong, and the difference may matter for how capital and labor are allocated across firms.  Yet the beliefs that managers hold about how technologies work, and the extent to which these beliefs vary across firms, are largely unobserved.

In this article, we develop a method for extracting the causal theories that managers articulate about new technologies during quarterly earnings conference calls.  A causal theory, as we define it, consists of two components: an upstream cause (``we are investing in cloud computing because our customers demand it'') and a downstream effect (``cloud computing will reduce our operating costs by twenty percent'').  We apply this method to 450,095 earnings-call transcripts from 19,469 firms in 94 countries, covering the period 2002--2024 and 29 disruptive technologies identified by \citet{Kalyanietal2025}.  Using a chain-of-thought large language model, we extract 187,647 causal snippets containing 197,818 upstream cause spans and 385,418 downstream effect spans.  We organize these into interpretable categories using a two-layer taxonomy: a first layer of up to 15 technology-specific cause and effect categories per technology and a second layer of 5 aggregate cause and 5 aggregate effect categories that are comparable across technologies, firms, industries, and time.  The resulting data set is a firm--technology--quarter panel of the causal frames managers use when discussing new technologies.

We use this data set to establish four main findings.

First, the causal theories managers hold about technology are heterogeneous at the firm level.  At the aggregate level, the cross-technology variation in cause and effect compositions lines up with economic priors: technologies such as hybrid vehicles and solar power, where adoption depends on policy mandates and subsidies, show high shares of regulatory causes; technologies such as machine learning, where adoption requires building new capabilities, show high shares of technology innovation causes; consumer-facing technologies such as mobile payments and online streaming are dominated by demand-related causes.  After absorbing technology-by-time and industry-by-time fixed effects, however, a substantial share of the variation in cause and effect shares remains.  Managers in the same industry, engaging with the same technology at the same time, articulate different theories about why the technology matters and what it will deliver.  This finding parallels the observation in \citet{Hassanetal2019} that most of the variation in firm-level political risk is at the firm rather than the aggregate or sector level.  But the object being measured here is different: not how much attention a firm pays to a phenomenon, but the composition of its causal reasoning about that phenomenon.

Second, these theories are persistent and portable across technologies.  Because we observe the same firm discussing multiple technologies, we can estimate firm fixed effects in cause and effect shares after removing technology-specific and time-varying components.  A firm whose cloud-computing narrative emphasizes market demand as the adoption driver also tends to frame machine learning in demand-pull terms, even when the two technologies have different adoption patterns in the cross-section.  When a firm begins discussing a technology for the first time, its initial causal frame resembles the frame it uses for technologies already in its portfolio.  This cross-technology persistence is consistent with models in which agents form beliefs by overweighting the most salient features of past experience (\citealt{GennaioliShleifer2010}; \citealt{BordaloGennaioli2018}).

Third, causal theories predict firm actions.  Conditional on the same technology, industry, and time period, firms whose managers frame technology effects in terms of revenue expansion and market growth invest more, pursue more technology-related acquisitions, and hire more.  Firms whose managers frame effects in terms of cost reduction and operational efficiency are more likely to restructure and reduce headcount.  The identifying variation comes from within technology--industry--time cells.  We further exploit the distinction between management's prepared remarks and analysts' questions during the Q\&A session: when the analyst's causal frame diverges from the manager's, subsequent analyst forecast dispersion is higher and absolute forecast errors are larger.

Fourth, beliefs that deviate from ex post benchmarks are associated with worse outcomes.  For each technology, we construct an ex post benchmark: the average causal frame that prevails several years after initial adoption, once early uncertainty has resolved. Firms whose early causal theories deviate more from this benchmark experience worse subsequent performance, measured by investment efficiency and real outcomes such as technology-related job creation and patenting.  The cross-technology spillover in beliefs provides additional traction.  When a firm's initial causal frame for a new technology looks like a carryover from the technologies the firm already discusses, rather than reflecting the new technology's characteristics, outcomes tend to be worse.  These results are consistent with the view that heterogeneous beliefs about technology contribute to the dispersion of firm outcomes, a channel distinct from the financial frictions and policy distortions studied in the misallocation literature (\citealt{HsiehKlenow2009}).

Relative to earlier text-based measures of firm-level exposure (\citealt{Hassanetal2019}; \citealt{Sautneretal2023}; \citealt{Hassanetal2023RFS}), the main measurement contribution of this article is that we extract the \emph{structure} of beliefs rather than their level or direction.  Existing measures quantify how much attention a firm pays to a topic, and sometimes whether the sentiment is positive or negative.  Two firms can devote the same amount of earnings-call time to artificial intelligence yet hold different theories about why AI matters and what it will do.  Our method captures this difference.  A second advantage of our setting is that we observe the same firm reasoning about different technologies at the same point in time.  This within-firm, across-technology variation allows us to separate managerial belief styles from technology-specific factors.

We note three main caveats to our analysis.  First, all of our measures likely contain measurement error and should be interpreted with caution.  Earnings-call language is shaped in part by strategic disclosure incentives, and causal statements may reflect what managers wish to communicate rather than what they truly believe.  We address this concern by comparing causal frames in the scripted management presentation with those in the less controlled Q\&A section; the main patterns hold in both, though the Q\&A-based measures are noisier.  Second, causal talk may partly reflect the technology's lifecycle rather than belief heterogeneity: all firms may discuss causes early and effects late.  We find that this within-firm-technology trajectory is real, but the economically interesting variation is in the \emph{composition} of causes and effects, not just their relative frequency.  Two firms can both be in the early phase of the same technology but differ sharply in which causes they emphasize, and these compositional differences predict different actions and outcomes.  Third, all of our measures should be interpreted as indicative of beliefs as perceived and articulated by managers and call participants.  These perceptions may differ from the true data-generating process of technology adoption.\footnote{A growing literature studies how managers' expectations affect firm actions, even when they are biased \citep{GennaioliShleifer2018}.}

Our work builds on several strands of prior literature.  A large literature studies the diffusion of technology, from \citet{Griliches1957} onward.  This literature has focused on diffusion patterns using physical adoption proxies and on the forces that make diffusion faster or slower (e.g., \citealt{Mansfield1961}; \citealt{AcemogluLinn2004}; \citealt{Hall2006}).  Most recently, \citet{Kalyanietal2025} identify hundreds of new technologies through textual analysis of patents, job postings, and earnings calls, and document stylized facts about the geographic and skill-level diffusion of related jobs.  We contribute to this literature by providing a measurement of \emph{why} firms say they adopt each technology and what they expect it to deliver, a dimension of diffusion that is distinct from the intensity and speed of adoption.

A second strand studies managerial beliefs and their economic effects.  This literature has shown that managers' subjective expectations affect investment (\citealt{Malmendier2005}), financing decisions (\citealt{BenDavid2013}), and hiring (\citealt{Coibion2018}).  Theoretically, \citet{GennaioliShleifer2010} and \citet{BordaloGennaioli2018} develop models in which agents overweight representative features of their experience when forming beliefs, producing predictable patterns of over- and underreaction.  We contribute by providing a revealed-preference measure of the structure of managerial beliefs about technology, extracted at scale from a high-scrutiny disclosure setting.  The cross-technology portability of causal frames that we document provides new evidence that the cognitive patterns studied in the diagnostic expectations literature extend to how managers reason about technology.

A third strand studies misallocation of resources across firms.  \citet{HsiehKlenow2009} and \citet{Bloometal2018} document large productivity differences across firms within narrowly defined industries and explore the role of financial frictions, policy distortions, and management practices.  We document a new channel: firms whose managers hold causal theories about technology that turn out to be at odds with the technology's actual trajectory misallocate resources toward or away from the technology, with measurable consequences for performance.

The remainder of the article is organized as follows.  Section~II describes the data and measurement approach.  Section~III validates the measures.  Section~IV characterizes the structure of managerial beliefs.  Section~V links beliefs to firm actions.  Section~VI tests whether biased beliefs predict misallocation.  Section~VII concludes.


% ============================================================
\section*{II. Data and Measurement}
\label{sec:data}
%Here it is important to understand and provide good reasons for any grouping that we do POST-TNT-LLM--we also need to start thinking about how to audit this stuff
\note{This section describes (a)~the earnings-call corpus, (b)~the LLM-based causal-triple extraction pipeline, (c)~the two-layer TNT-LLM taxonomy construction, and (d)~the resulting panel.  Follow closely the methodology in the Chai (2025) presentation (slides 9--15).  Key subsections:}

\subsection*{II.A. Earnings Conference Call Data}
%We need to be precise about which technology data we are using; the working paper version or the published paper's. Please furnish the details so this can be made clear and ensure that the cites are consistent (e.g. dataset to wp and conceptual/findings to the qje)
\note{Describe the corpus: 450,095 calls, 2002--2024, 19,469 firms, 94 countries, from Refinitiv Eikon.  Standard sample filters.  The 29 technologies come from Kalyani et al.\ (2025, QJE).  Describe keyword-based filtering and the 7-sentence context window.  Follow the level of detail in Hassan et al.\ (2019, Section II), which describes sample coverage, call structure, and data sources in compact paragraphs.}

\subsection*{II.B. Extracting Causal Statements}
%start thinking about audit procedures 
\note{Detail the four-step LLM pipeline: (1)~technology relevance check, (2)~causal content check, (3)~extraction of upstream causes and downstream effects, (4)~normalization into triples with promotes/suppresses direction.  Give concrete examples (slide~11).  Report yield: 187,647 causal snippets, 197,818 cause spans, 385,418 effect spans.  This is the analog of the bigram-counting methodology in Hassan et al.\ (2019, Section III.A) but is more involved, so describe each step precisely and refer the reader to the Online Appendix for prompt text and LLM parameters.}

\subsection*{II.C. Taxonomy Construction}

\note{Describe the two-layer TNT-LLM approach.  Layer~1: technology-specific taxonomies ($\leq$15 cause and $\leq$15 effect categories per technology).  Layer~2: aggregation into 5 macro cause and 5 macro effect categories. Present the 5+5 macro categories in a table (Table~I) with brief definitions and shares.  Full technology-specific taxonomies go in the Online Appendix. The economic rationale for the macro categories should be stated briefly: the five causes map onto standard drivers of technology adoption (supply push, demand pull, policy, strategic choice, cost), and the five effects map onto standard outcomes (growth, cost savings, market share, innovation, efficiency).}

\subsection*{II.D. Panel Construction}

The unit of observation in our analysis is a firm--technology--quarter triple $(i,k,t)$.  For each observation, we construct the following variables from the extracted causal statements described in Sections~II.B and~II.C.

Let $n^{\text{cause}}_{ikt,c}$ denote the number of cause spans assigned to macro category $c \in \{1,\ldots,5\}$ for firm $i$, technology $k$, in quarter $t$, and let $N^{\text{cause}}_{ikt} = \sum_{c=1}^{5} n^{\text{cause}}_{ikt,c}$ denote the total number of cause spans.  We define the \emph{cause share} for category $c$ as
\begin{equation}
\label{eq:cause_share}
s^{\text{cause}}_{ikt,c} = \frac{n^{\text{cause}}_{ikt,c}}{N^{\text{cause}}_{ikt}}.
\end{equation}
The five \emph{effect shares} $s^{\text{effect}}_{ikt,c}$ are defined analogously, where $N^{\text{effect}}_{ikt} = \sum_{c=1}^{5} n^{\text{effect}}_{ikt,c}$; in both cases $c$ indexes the five macro categories on the relevant side, with the mapping from index to category name differing between causes and effects as shown in Table~\ref{tab:macro-categories}.  By construction, the five cause shares sum to one for each observation, as do the five effect shares.  We also compute the \emph{total span count} $N_{ikt} = N^{\text{cause}}_{ikt} + N^{\text{effect}}_{ikt}$ and the \emph{cause-to-effect ratio} $N^{\text{cause}}_{ikt} / N^{\text{effect}}_{ikt}$, which serve as measures of the intensity and stage of a firm's engagement with a technology.

To ensure that the share variables are well defined, we impose a minimum-evidence threshold: we require at least three spans on a given side (cause or effect) for an observation to enter the estimation sample.  We first deduplicate at the snippet level, so that each unique causal statement contributes at most one span per macro category.  We then collapse the span-level data to firm--technology--quarter counts, compute shares, and drop observations below the threshold.  Table~\ref{tab:summary-stats} reports summary statistics for the resulting panel.

The estimation sample contains 13,723 firm--technology--quarter observations spanning 2,812 unique firms, all 29 technologies, and 92 calendar quarters (2002:Q1--2024:Q4) across 42 countries.  The median observation draws on 12 spans in total, with medians of 5 on the cause side and 7 on the effect side; the mean cause-to-effect ratio is 0.80, indicating that managers articulate somewhat more downstream effects than upstream causes.  Panels~C and~D of Table~\ref{tab:summary-stats} report the distribution of observation-level cause and effect shares.  On the cause side, Technology Innovation and Advancement (mean share 0.311) and Market Demand and Consumer Behavior (0.297) together account for over 60\% of the typical observation's causal narrative, while Regulatory and Policy Drivers is the least prevalent cause category (0.064).  The effect side is more evenly distributed, led by Market Expansion and Adoption (0.304) and Revenue and Financial Growth (0.243).  These observation-level means differ slightly from the pooled span-level shares in Table~\ref{tab:macro-categories}, which weight each span equally regardless of observation size.

Because many firms discuss more than one of the 29 technologies during our sample period, the panel is not restricted to one observation per firm--quarter.  A firm that discusses both cloud computing and machine learning in the same quarter contributes two observations.  The typical firm contributes a modest number of observations---the median is two and the mean is 4.9---reflecting heterogeneity in both the breadth of technologies discussed and the duration of sample participation.  This within-firm, across-technology variation is a distinguishing feature of our data: it allows us to include firm fixed effects while still estimating the role of technology-specific belief composition, and it enables the cross-technology belief spillover tests in Section~IV.B.  We verify that our main results are robust to collapsing to the firm--quarter level by averaging shares across technologies (Online Appendix Table~XX).%TODO: replace XX with actual table number

\subsection*{II.E. Speaker Identification}
%We dont need this. However, we probably do need to know who made the causal statement to refine potential tests -- not sure whether this paper should be about individual managers or about the theoretical models of the tmt. Separating out Q\&A from presentation is too basic to be discussed in a subsection
\note{Describe how prepared remarks are separated from Q\&A, and within Q\&A how manager responses are distinguished from analyst questions.  Define the disagreement variable: distance between the manager's cause/effect share vector and the analyst's cause/effect share vector (e.g., cosine distance or Euclidean distance in the 5-dimensional share space).  This parallels the management-presentation vs.\ Q\&A distinction in Hassan et al.\ (2019, Section III) and Sautner et al.\ (2023).}


% ============================================================
\section*{III. Validation}
\label{sec:validation}
%We probably need to make this more interesting than the validation stuff we did in prior work, shorten it (III.A-D) and move to appendix. Would be good to think about what audiences would find hard to believe and see if we can come up with ways to show evidence to reassure them
\note{This section establishes that the measures capture what we claim.  Follow the validation logic of Hassan et al.\ (2019, Section IV) and Sautner et al.\ (2023, Section III): face validity of top-scoring transcripts, intuitive cross-sectional and time-series variation, correlation with external proxies, measurement error quantification.}

\subsection*{III.A. Top-Scoring Transcripts}

We begin the validation by examining whether the LLM-assigned macro categories correspond to the actual content of the underlying transcript text.
Following the face-validity exercises in \citet{Hassanetal2019} and \citet{Sautneretal2023}, we present in Online Appendix Table~\ref{tab:top-scoring} one \textit{Technology Innovation and Advancement} (``tech-push'') excerpt and, where available, one \textit{Market Demand and Consumer Behavior} (``demand-pull'') excerpt for each of the 29 technologies.
We select only excerpts in which the extracted causal phrase appears verbatim in the surrounding sentence window, so that the reader can verify the category assignment directly.

The contrast is immediately visible across technologies.
For fracking, the tech-push excerpt describes ``technology advances in horizontal drilling and hydraulic fracturing techniques''---engineering innovation---while the demand-pull excerpt discusses ``sustained demand for hydraulic fracturing applications in the energy sector.''
For cloud computing, the tech-push excerpt references ``advanced capabilities for continuously monitoring thousands of Kafka clusters,'' whereas the demand-pull excerpt highlights ``growing needs from professional asset and investment managers for data analytical tools.''
For solar power, the tech-push excerpt describes ``Trimetasphere carbon nanomaterials technology'' aimed at improving organic solar cell efficiency, while the demand-pull excerpt describes ``robust demand for power supply equipment and PCS for mega solar power generation projects in Japan.''
In each case, the same technology is discussed through qualitatively different causal lenses, and the taxonomy captures this distinction.

Coverage is comprehensive: all 29 technologies have a clear tech-push excerpt, and 27 of 29 have a demand-pull excerpt (the exceptions---computer vision and drug conjugates---lack cause-side spans classified as Market Demand).
Section~III.E provides a systematic assessment of measurement error.

\subsection*{III.B. Cross-Technology Variation}

Figure~\ref{fig:cross-tech} reports the share of each macro cause category (Panel~A) and macro effect category (Panel~B) for every technology in the sample, pooled across firms and quarters.
The taxonomy is constructed by the LLM without knowledge of each technology's economic context; the question is whether the resulting category shares recover the adoption drivers that domain knowledge would predict.

On the cause side, the cross-technology variation aligns with economic priors.
Technologies whose adoption is widely understood to be supply-driven show the largest shares of Technology Innovation and Advancement: drug conjugates, virtual reality, computer vision, and machine learning are dominated by this category, reflecting the central role of R\&D breakthroughs and platform capabilities in their diffusion.
At the other end of the spectrum, technologies with strong demand-side narratives---mobile payments, social networking, wireless charging---show large shares of Market Demand and Consumer Behavior.
Solar power and hybrid electric vehicles stand out for their substantial Regulatory and Policy Drivers shares, consistent with the well-documented role of government mandates and subsidies in clean-energy adoption.
Fracking---whose spread depends on both drilling innovation and commodity economics---exhibits a characteristic mix of Technology Innovation and Cost and Economic Viability.

The effect side exhibits similarly interpretable variation (Panel~B).
Consumer-facing technologies such as mobile payments and social networking are dominated by Market Expansion and Adoption effects, reflecting growth-oriented expectations associated with network goods.
R\&D-intensive technologies such as drug conjugates, computer vision, and solar power show large Product and Service Innovation shares.
Technologies deployed primarily for operational improvement---fracking, RFID tags---have above-average Cost Reduction and Efficiency shares.
These cross-technology patterns parallel the sector-level variation checks in \citet{Hassanetal2019} and the emissions-correlation checks in \citet{Sautneretal2023}: the taxonomy captures economically meaningful differences without being told what to look for.
Online Appendix Figures~\ref{fig:app-pertech-cause} and~\ref{fig:app-pertech-effect} present the corresponding technology-specific time series of category shares.

\subsection*{III.C. Time-Series Patterns}

Sections~III.A and~III.B validate the cross-sectional content of the measures; we now examine their time-series dynamics.
Figure~\ref{fig:time-series} plots the number of unique firms with at least one causal snippet about a given technology per quarter for six technologies spanning three diffusion archetypes.

Panel~A presents two platform technologies---cloud computing and machine learning/AI---that exhibit a sustained ramp in discussion intensity.
Cloud computing enters the earnings-call vocabulary around 2008 and climbs steadily through 2022, consistent with the gradual enterprise migration to cloud infrastructure.
Machine learning/AI follows a similar but later trajectory, accelerating after 2016 as deep-learning applications reach production scale, with a further sharp increase in late 2022 coinciding with the release of large language models.
Neither technology displays the transient spike characteristic of a fad; instead, each embeds itself progressively in the managerial conversation.

Panel~B illustrates the opposite pattern for two feature technologies.
RFID tags spike in the mid-2000s---peaking shortly after Walmart's 2003 supplier mandate catalyzed industry attention---then decline as supply-chain tagging matures into routine practice.
Touch screens follow a similar arc on a smaller scale, with discussion rising after the 2007 iPhone launch and fading as capacitive displays become a commodity input.
In both cases, the measures capture the adoption-and-decline cycle without any supervision.

Panel~C shows policy-driven technologies---solar power and hybrid electric vehicles---whose discussion intensity moves in episodic waves tied to identifiable policy events.
Solar power discussion surges after the American Recovery and Reinvestment Act (2009), contracts as stimulus funding expires, revives around the Paris Agreement and the solar investment tax credit extension (2015), dips following the imposition of solar tariffs (2018), and rebounds with the Inflation Reduction Act (2022).
Hybrid electric vehicles trace a more subdued but temporally similar path.
The correspondence between policy milestones and discussion intensity provides external validation that the measures respond to real-world adoption incentives rather than noise.

Taken together, the three panels confirm that the discussion-intensity measures capture qualitatively distinct diffusion dynamics across technology types.
The within-quarter firm-level variation that drives the analyses in Sections~IV--VI sits on top of these aggregate patterns.

\subsection*{III.D. Correlation with External Proxies}

\note{Correlate measures with: (1)~Kalyani et al.\ (2025) technology mention counts in job postings, (2)~green job postings and green patents from Sautner et al.\ (2023) for climate-related technologies, (3)~R\&D expenditure, (4)~patent filings in relevant technology classes.  The measures should correlate positively with these proxies but contain significant incremental information (the composition of causation, not just the intensity of discussion).  Table~III.}

\subsection*{III.E. Measurement Error}
%This might be the only subsection that we report in the main text
\note{Follow the AR(1) instrumental-variable approach from Hassan et al.\ (2019, Section V.D) and Sautner et al.\ (2023, Table VI).  Back out the implied share of variation attributable to measurement error.  Use an alternative text source (e.g., 10-K MD\&A filings) as the instrument.  Target: implied measurement error of similar order as in the political risk and climate exposure papers (5--10\%).  Table~IV.}


% ============================================================
\section*{IV. The Structure of Managerial Beliefs}
\label{sec:structure}

\note{This is the main descriptive contribution.  The goal is to establish three properties of managerial beliefs about technology: (1)~most variation is at the firm level, (2)~beliefs are persistent and spill over across technologies, (3)~belief dynamics are consistent with diagnostic expectations.  This section parallels the ``Firm-Level Political Risk'' section (Section V) of Hassan et al.\ (2019), which was the paper's key descriptive contribution.}


\subsection*{IV.A. Variance Decomposition}

A feature of the associations documented in Section~\ref{sec:validation} is that they hold even when we condition on technology and time fixed effects.  This finding may be unexpected if one views technology adoption as driven primarily by aggregate trends or by industry-specific shocks that affect all firms in a sector at roughly the same time.  To assess the relative contributions of aggregate, industry, technology, and firm-level variation to the measures, we conduct a simple analysis of variance, asking how much of the variation in cause and effect shares is accounted for by various sets of fixed effects.

For each of the five macro cause shares and five macro effect shares defined in Section~II.C, we project the share on successive sets of fixed effects and record the incremental $R^2$ from each addition.  We then average the incremental $R^2$ values across the five cause shares and, separately, across the five effect shares.  Table~\ref{tab:var_decomp_cause} reports the results for cause shares and Table~\ref{tab:var_decomp_effect} for effect shares.  We present results at three levels of industry granularity: two-digit SIC (column~1), three-digit SIC (column~2), and four-digit SIC (column~3).

Consider first the cause shares (Table~\ref{tab:var_decomp_cause}).  Time fixed effects account for 1.89\% of the variation at the two-digit SIC level.  Industry fixed effects add 4.70\%, and industry-by-time interactions add another 6.32\%.  Technology fixed effects contribute 13.08\% and technology-by-time interactions 6.92\%.  Taken together, these observable dimensions account for about a third of the total variation.  The remaining 67.09\% is at the firm level.  Of this firm-level component, 18.55 percentage points reflect permanent differences across firms within industries (firm fixed effects) and 48.54 percentage points reflect time-varying, firm-specific variation (the residual).

As we move to finer industry definitions, the industry and industry-by-time components naturally absorb more variation.  At the four-digit SIC level, industry-by-time interactions account for 12.37\%, up from 6.32\% at the two-digit level.  But the firm-level component remains at 60.02\%, of which 16.39 percentage points are permanent and 43.63 are time varying.  Even at this level of granularity, the majority of the variation in what managers say drives technology adoption cannot be attributed to the technology, the industry, or the time period.

The effect shares (Table~\ref{tab:var_decomp_effect}) show a similar pattern but with a somewhat larger firm-level component.  At the two-digit level, firm-level variation accounts for 73.62\% of the total; at four-digit, 67.15\%.  The permanent firm component is again stable across industry definitions, ranging from 16.13\% to 18.34\%.  The technology and technology-by-time components are slightly smaller for effect shares (about 15--16\% combined, compared with about 20\% for cause shares).  This difference is interpretable: why a technology gets adopted is more technology-determined (solar power really is adopted for different reasons than cloud computing), while what firms expect a technology to deliver depends more on the firm's own strategic orientation and the uses it has in mind.\footnote{These numbers can be compared with analogous decompositions in related work.  \citet{Hassanetal2019} report that 91.69\% of the variation in firm-level political risk is at the firm level at two-digit SIC granularity, falling to 78.33\% at four-digit SIC (their Table~VIII).  \citet{Sautneretal2023} report firm-level components of 70--96\% across their climate change exposure measures (their Table~V).  Our numbers are at the lower end of this range, which is expected: technology-specific factors absorb 11--16\% of the variation in our measures, whereas political risk and climate exposure lack a comparable technology dimension.  Once this technology component is accounted for, the remaining pattern is similar: firm-level variation dominates industry and time effects.}

Two concerns about this finding merit discussion.  First, part of the firm-level variation might reflect measurement error rather than true heterogeneity in beliefs.  We address this concern in Section~III.E, where we estimate the share of variation attributable to measurement noise using the approach in \citet{Hassanetal2019}.  The associations between our measures and corporate outcomes documented in Sections~V and~VI provide further evidence that the firm-level variation has economic content.  Second, some firm-level variation may reflect differences in objective circumstances that are not captured by industry-by-time interactions, such as differences in customer mix or competitive positioning.  We cannot fully rule out this possibility, but the stability of the permanent firm component across levels of industry granularity suggests that at least part of the variation reflects persistent differences in how managers reason about technology, rather than transient differences in circumstances.


\subsection*{IV.B. Cross-Technology Belief Spillover}

The variance decomposition establishes that most variation in cause and effect shares is at the firm level.  A natural question is whether this firm-level component reflects a portable cognitive style---a tendency to reason about technology in a particular way---or technology-specific idiosyncrasy.  We test this by asking whether a firm's causal frame for one technology predicts its causal frame for another.

For each firm--technology--quarter observation $(i,k,t)$, we construct a leave-one-out peer share: the firm's average share in category $c$ across all its other technologies $j \neq k$, pooled over all quarters.  We suppress the cause/effect superscript where the side is clear from context.  The peer share for firm $i$, technology $k$, category $c$ is
\begin{equation}
\label{eq:peer_share}
\bar{s}^{\neg k}_{i,c} = \frac{1}{|\mathcal{K}_i \setminus \{k\}|} \sum_{j \in \mathcal{K}_i \setminus \{k\}} \bar{s}_{ij,c}
\end{equation}
where $\mathcal{K}_i$ is the set of technologies discussed by firm $i$ and $\bar{s}_{ij,c}$ is firm $i$'s time-averaged share in category $c$ for technology $j$.  Because the peer share averages over all quarters, it is constant within a firm--technology--category cell and does not vary over time.  We then estimate a stacked regression with one row per firm--technology--quarter--category:
\begin{equation}
\label{eq:spillover}
s_{ikt,c} = \beta \, \bar{s}^{\neg k}_{i,c} + \alpha_{k \times c} + \gamma_{t \times c} + \varepsilon_{ikt,c}
\end{equation}
where $\alpha_{k \times c}$ and $\gamma_{t \times c}$ are technology$\times$category and quarter$\times$category fixed effects, and standard errors are clustered at the firm level.  A positive $\beta$ indicates that firms carry their causal frame across technologies: a firm that emphasizes, say, regulatory drivers when discussing solar power also emphasizes regulatory drivers when discussing electric vehicles, relative to other firms discussing the same technology in the same quarter.

Table~\ref{tab:spillover} reports the results for the 787 firms that discuss at least two technologies (7,390 firm--technology--quarter observations; 36,950 stacked rows).  Panel~A presents the cause-side estimates.  In the baseline specification (column~I), the coefficient on the peer share is 0.065 ($p < 0.01$): a 10-percentage-point higher peer cause share is associated with a 0.65-percentage-point higher own cause share.  The coefficient attenuates to 0.040 ($p < 0.05$) when we replace quarter fixed effects with industry$\times$quarter fixed effects (column~II), absorbing industry-specific time trends, and remains stable at 0.042 ($p < 0.05$) when we further add technology$\times$quarter fixed effects to absorb technology-specific trends (column~III).

Column~IV restricts the sample to the first quarter in which each firm discusses a given technology (1,925 observations before stacking).  This is the most direct test of belief portability: when a firm encounters a new technology for the first time, does its initial causal frame resemble its existing portfolio?  The coefficient is 0.049 ($p < 0.05$), indicating that the answer is yes.  The initial frame a manager brings to a new technology is shaped by the frames already in place for the firm's existing technologies.

Panel~B repeats the analysis for effect shares.  The spillover coefficients are larger and more precisely estimated throughout: 0.099 in the baseline (column~I), 0.103 with the most demanding fixed effects (column~III), and 0.087 at first appearance (column~IV).  All estimates are significant at the 1\% level.  The stronger effect-side spillover is consistent with the variance decomposition, which showed a larger firm-level component for effect shares (67--74\%) than for cause shares (60--67\%).  Managers appear to have particularly portable expectations about what technologies will deliver, even more so than about why technologies are being adopted.

Online Appendix Table~\ref{tab:spillover_robust} presents four robustness checks.  The spillover result strengthens when restricting to firms with at least three technologies (cause $\hat{\beta} = 0.147$, effect $\hat{\beta} = 0.160$, both $p < 0.01$), survives snippet-count weighting, is stable when any single category is dropped from the stacked regression, and does not vary significantly with the cosine similarity between the focal and peer technologies' cause-share profiles.

The stacked regression imposes a common spillover coefficient across the five macro categories.  Online Appendix Table~\ref{tab:spillover_by_cat} relaxes this restriction by estimating the spillover regression separately for each category.  Portability varies substantially.  On the cause side, Technology Innovation and Advancement is the most portable category ($\hat{\beta} = 0.129$, $p < 0.01$), while Market Demand, Regulatory Drivers, and Cost and Economic Viability show small, insignificant coefficients.  On the effect side, Revenue and Financial Growth ($\hat{\beta} = 0.191$, $p < 0.01$) and Operational Efficiency ($\hat{\beta} = 0.132$, $p < 0.01$) are the most portable, while Market Expansion and Adoption is not portable at all.  The pattern is economically interpretable: the categories that reflect general strategic orientation---supply-push versus demand-pull reasoning on the cause side, growth versus efficiency expectations on the effect side---transfer across technologies, while categories tied to technology-specific circumstances do not.  Notably, the two most portable effect categories correspond to the growth and efficiency orientations whose real consequences we examine in Section~V.

\subsection*{IV.C. Belief Dynamics}

Sections~IV.A and~IV.B establish that firm-level variation in belief composition is large and portable across technologies.  A remaining question is whether these belief structures are persistent over time.  If the causal frames documented above were transient---reflecting one-off phrasing or short-lived attention---they would be unlikely to drive the firm outcomes we examine in Section~V.  We now show that initial frames anchor beliefs for years.

For each firm--technology pair $(i,k)$ observed in at least two quarters, we define the initial share $s_{ik0,c}$ as the category-$c$ share from the first quarter in which firm~$i$ discusses technology~$k$.  We then estimate a stacked persistence regression on all subsequent observations:
\begin{equation}
\label{eq:persistence}
s_{ikt,c} = \beta \, s_{ik0,c} + \alpha_{k \times c} + \gamma_{t \times c} + \varepsilon_{ikt,c}
\end{equation}
where $\alpha_{k \times c}$ and $\gamma_{t \times c}$ are technology$\times$category and quarter$\times$category fixed effects, and standard errors are clustered at the firm level.  A positive $\beta$ indicates that a firm's initial frame for a technology predicts its frame in later quarters, conditional on what other firms discussing the same technology at the same time think.  The sample consists of 1,933 firm--technology pairs observed in at least two quarters (9,773 post-initial observations; 48,865 stacked rows).

Table~\ref{tab:persistence} reports the results.  In the pooled specification (column~I), the coefficient on the initial share is 0.147 for cause shares and 0.172 for effect shares (both $p < 0.01$).  A 10-percentage-point higher initial cause share predicts a 1.5-percentage-point higher current share, absorbing technology and quarter trends.  Columns~(II) through~(IV) split the sample by tenure horizon.  In the first year after initial mention (column~II), persistence is strongest: $\hat{\beta} = 0.257$ for causes and 0.268 for effects.  At the medium horizon of one to three years (column~III), the coefficients attenuate to 0.151 and 0.196 but remain highly significant.  Even at horizons beyond three years (column~IV), with 24,385 stacked observations, the initial frame still predicts the current frame: $\hat{\beta} = 0.101$ for causes and 0.115 for effects ($p < 0.01$).

Figure~\ref{fig:persistence} traces this decay quarter by quarter.  The average cross-category correlation between initial and current shares starts near 0.48 at one-quarter tenure and declines gradually, stabilizing around 0.25--0.30 at five or more years.  The decline is smooth rather than abrupt, suggesting slow partial updating rather than discrete regime changes.  The cause-to-effect ratio itself does not exhibit a systematic lifecycle pattern---early engagement is not cause-heavy relative to later engagement---but the \emph{composition} of both cause and effect beliefs shows durable persistence.

Two additional patterns merit discussion.  First, effect beliefs are more persistent than cause beliefs at all horizons.  The within-pair drift---measured as the average Euclidean distance between the current and initial share vectors across all post-initial quarters---is 0.549 for cause shares versus 0.472 for effect shares, with cause drift exceeding effect drift in 64.5\% of firm--technology pairs.  This asymmetry is consistent with the cross-technology spillover results: effect expectations---what technologies will deliver---are both more portable across technologies (Section~IV.B) and more stable over time.  The expectations about growth versus efficiency that we examine in Section~V appear to be a particularly durable feature of the firm's cognitive style.

Second, the persistence documented here is not mechanical.  The technology$\times$category and quarter$\times$category fixed effects absorb any technology-level trends, so the coefficient captures within-technology, within-quarter persistence of the firm's idiosyncratic frame.  A firm that initially over-emphasizes technology innovation as a cause of cloud computing adoption (relative to other firms discussing cloud computing that quarter) continues to over-emphasize it years later.  This durability is what makes the belief-to-action predictions in Section~V credible: the causal frames that predict differential investment, hiring, and restructuring are not transient noise but persistent features of how the firm reasons about technology.

\subsection*{IV.D. Manager--Analyst Disagreement}
%Again, not sure whether we want to do this
\note{Define and characterize disagreement between the manager's causal frame (prepared remarks and manager responses in Q\&A) and the analyst's causal frame (analyst questions).  Show that disagreement varies across firms and technologies and is highest early in a technology's lifecycle and for technologies where the causal structure is ambiguous (e.g., AI).  This subsection sets up the predictive tests in Sections~V and~VI.}


% ============================================================
\section*{V. Beliefs and Firm Actions}
\label{sec:actions}

\note{Show that causal theories predict what firms do, conditional on the same technology, industry, and time.  The identification relies on within technology--industry--time variation.  Outcomes: investment, R\&D, employment, M\&A, patenting.  The contribution is showing that belief composition matters above and beyond belief intensity (total snippet count).  This section parallels the ``Effects'' portions of Hassan et al.\ (2019, Tables IV--VI), Sautner et al.\ (2023, Tables VII--VIII), and Hassan et al.\ (2023 RFS, Table 9).}

\subsection*{V.A. Growth-Oriented versus Efficiency-Oriented Theories}

\note{Define two summary measures on the effects side: (1)~growth orientation = share of Revenue Growth + Market Expansion + Product Innovation, (2)~efficiency orientation = share of Cost Reduction + Operational Efficiency.  Show that firms with higher growth orientation invest more, hire more, and pursue more acquisitions; firms with higher efficiency orientation restructure more.  Include firm FE, technology $\times$ time FE, and industry $\times$ time FE.  Control for total snippet count to separate framing from engagement.  Table~VI.}

\subsection*{V.B. Cause-Side Theories and Strategic Choices}

\note{The causes a manager emphasizes should predict how the firm pursues the technology.  Firms emphasizing Market Demand invest in marketing and distribution.  Firms emphasizing Technology Innovation invest in R\&D and hire technical talent.  Firms emphasizing Strategic Partnerships pursue M\&A and joint ventures.  This is a more granular test, may require technology-specific outcome variables (patent filings, technology-related job postings from Burning Glass/Lightcast).  Table~VII.}

\subsection*{V.C. Manager--Analyst Disagreement and Market Outcomes}
%not sure whether we should go in this direction
\note{Show that manager--analyst disagreement on the causal frame predicts: (a)~higher analyst forecast dispersion, (b)~larger absolute forecast errors, (c)~higher subsequent stock return volatility.  Control for total technology exposure and attention level.  Table~VIII.  The interpretation: when the manager's theory diverges from the market's theory (proxied by the analyst), uncertainty is higher.  \textbf{Challenge:} distinguishing genuine belief disagreement from strategic obfuscation.  The Q\&A-based measures help because the analyst forces the manager to respond to an alternative theory.}


% ============================================================
\section*{VI. Beliefs and Misallocation}
\label{sec:misallocation}
%This section is what econ journals care about--evidence the beliefs have market changing effects -- this needs to be as high-impact as we can go

\note{This is the main payoff.  Do biased beliefs cause predictably worse outcomes?  Two approaches.  This section should have the most novel economic content and the cleanest identification.  It parallels the policy-crises analysis in Hassan et al.\ (2019, Section VI.B) and the demand-vs.-supply shock analysis in Hassan et al.\ (2023 RFS, Section 4.3) in that it uses the measurement innovation to shed light on a specific economic mechanism.}

\subsection*{VI.A. Ex Post Benchmarking}

\note{For each technology, construct a benchmark causal frame: the average cause and effect shares that prevail 3--5 years after the technology's peak diffusion phase.  Measure the distance between a firm's early causal frame and this benchmark.  Test: does larger distance predict worse subsequent performance?  Outcomes: investment efficiency, technology-related job creation, patenting, risk-adjusted stock returns.  Table~IX.  \textbf{Challenge:} ``ex post correct'' is noisy.  The benchmark is an average; some deviating firms may have superior private information.  Show that, on average, deviation predicts worse outcomes.  Discuss this limitation directly (as Hassan et al.\ 2019 discuss measurement error caveats).}

\subsection*{VI.B. Cross-Technology Belief Transfer and Outcomes}

\note{When a firm begins engaging with a new technology, measure the extent to which its initial causal frame looks like a carryover from the firm's existing technology portfolio versus reflecting the new technology's characteristics.  Decompose the firm's initial cause-share vector into a component predicted by its existing belief style and a residual.  Firms with a larger carryover component should experience worse outcomes for the new technology.  Table~X.  \textbf{Challenge:} endogeneity.  Firms that rely on analogy may differ on unobservables (management quality, resources).  Control for firm characteristics; discuss IV strategies if plausible.  At minimum, show robustness to controlling for firm size, age, R\&D intensity, industry, and prior technology adoption intensity.}

\subsection*{VI.C. Aggregate Belief Cycles}

\note{Aggregate the firm-level results to the technology level.  When many managers simultaneously hold growth-oriented theories about a technology (high aggregate shares in revenue growth and market expansion), does the subsequent aggregate trajectory disappoint?  This is the technology-level analog of the Bordalo--Gennaioli--Ma--Shleifer credit-cycle overreaction: collective optimism predicts collective disappointment.  Conversely, when the consensus frame is efficiency-oriented, subsequent outcomes should exceed expectations.  This subsection is more speculative and may be moved to the Online Appendix if the data do not support it cleanly.  If the results hold, they connect the firm-level findings to a broader question about how belief heterogeneity in technology adoption affects the economy.}


% ============================================================
\section*{VII. Conclusion}
\label{sec:conclusion}

\note{Follow the compact conclusion style of Hassan et al.\ (2019, Section VII) and Kalyani et al.\ (2025, Section VIII): 3--4 paragraphs summarizing the main findings, restating the caveats, and pointing to directions for future research.  Avoid re-arguing the paper's importance.  Key points to cover:}

\note{(1) We construct a new measurement of the causal theories managers hold about technology.  The resulting panel covers 29 technologies, 22 years, and nearly 20,000 firms across 94 countries.  (2) We document that these theories are heterogeneous at the firm level, persistent across time, and portable across technologies.  (3) Beliefs predict actions: growth-oriented and efficiency-oriented theories lead to different resource allocation decisions conditional on the same technology, industry, and time.  (4) Beliefs predict mistakes: firms whose early causal frames deviate from the technology's ex post trajectory, and firms that rely on analogies from past technologies, experience worse outcomes.  (5) Restate the three caveats.  (6) Directions for future research: the panel can be applied beyond the 29 technologies studied here; the measurement approach can be extended to other domains where agents hold causal theories about complex phenomena (e.g., macroeconomic policy, climate adaptation).}



\clearpage
\input{Appendix/appendix}

\clearpage
{
\setstretch{0.7}
\bibliographystyle{chicago}
\bibliography{references}
}
\clearpage
\clearpage

\begin{figure}[p]
\caption{Cross-Technology Variation in Macro-Category Shares}
\label{fig:cross-tech}
\vspace{0.5em}
\centering
\textit{Panel A. Causes of Technology Adoption}\\[0.5em]
\includegraphics[width=\textwidth]{technology_macro_share_cause.png}
\end{figure}

\begin{figure}[p]
\ContinuedFloat
\caption[]{Cross-Technology Variation in Macro-Category Shares (continued)}
\vspace{0.5em}
\centering
\textit{Panel B. Effects of Technology Adoption}\\[0.5em]
\includegraphics[width=\textwidth]{technology_macro_share_effect.png}

\begin{flushleft}\footnotesize
\textit{Notes.} Each horizontal bar shows the composition of extracted cause spans (Panel~A) or effect spans (Panel~B) across the five macro categories for a given technology, pooled over all firms and quarters. Technologies are sorted by the share of the dominant category. The dashed line marks 80\%. See Table~\ref{tab:macro-categories} for macro-category definitions and Section~III.B for discussion.
\end{flushleft}
\end{figure}

\clearpage

\begin{figure}[p]
\caption{Time-Series Patterns of Technology Discussion Intensity}
\label{fig:time-series}
\vspace{0.5em}
\centering
\includegraphics[width=\textwidth]{figure_II_time_series.pdf}

\begin{flushleft}\footnotesize
\textit{Notes.} Each panel plots the number of unique firms with at least one
causal snippet about the indicated technology in a given quarter. Panel~A shows
platform technologies with a sustained ramp in discussion intensity. Panel~B
shows feature technologies with spike-and-decline patterns. Panel~C shows
policy-driven technologies with stop-go waves; dotted vertical lines mark the
American Recovery and Reinvestment Act (2009), the Paris Agreement and solar ITC
extension (2015), solar tariffs (2018), and the Inflation Reduction Act (2022).
Sample: 2002:Q1--2024:Q4. See Section~III.C for discussion.
\end{flushleft}
\end{figure}

\clearpage
\clearpage
\begin{table}[!htbp]\centering
\caption{Summary Statistics}
\label{tab:summary-stats}
\begin{flushleft}\footnotesize
\textit{Notes.} This table reports summary statistics for the firm--technology--quarter panel constructed from earnings conference call transcripts. Panel~A describes the technology snippet extraction from 443,478 text windows across 82,169 transcripts (2002--2024). Technology-relevant snippets are those where the LLM identifies a specific technology from the 29 disruptive technologies in \citet{kalyani2025}. The causal ratio is the share of technology-relevant snippets containing at least one causal statement. Panel~B reports dimensions of the causal panel after deduplication and applying a minimum threshold of three spans per side (cause or effect). Panels~C and~D report the distribution of cause and effect shares across the five macro categories; by construction, shares on each side sum to one for every observation. Panel~E reports intensity measures. See Section~II.D for variable definitions.
\end{flushleft}
\resizebox{\textwidth}{!} \\
Causal ratio & \multicolumn{5}{r}{67.2\%} \\
Unique transcripts & \multicolumn{5}{r}{82,169} \\
Unique technologies & \multicolumn{5}{r}{29} \\
Sample period & \multicolumn{5}{r}{2002--2024} \\
Snippets per transcript & 3.4 & 4.6 & 1.0 & 2.0 & 4.0 \\
\hline
\multicolumn{6}{l}{\textsc{Panel B. Causal Panel Dimensions}} \\
Firm--technology--quarter observations & \multicolumn{5}{r}{13,723} \\
Unique firms & \multicolumn{5}{r}{2,812} \\
Unique technologies & \multicolumn{5}{r}{29} \\
Unique quarters & \multicolumn{5}{r}{92} \\
Countries & \multicolumn{5}{r}{42} \\
Observations per firm & 4.9 & 7.9 & 1.0 & 2.0 & 5.0 \\
\hline
\multicolumn{6}{l}{\textsc{Panel C. Cause Shares}} \\
Technology Innovation \& Advancement & 0.311 & 0.286 & 0.000 & 0.286 & 0.500 \\
Market Demand \& Consumer Behavior & 0.297 & 0.271 & 0.000 & 0.256 & 0.500 \\
Regulatory \& Policy Drivers & 0.064 & 0.158 & 0.000 & 0.000 & 0.000 \\
Strategic Partnerships \& Investment & 0.194 & 0.241 & 0.000 & 0.100 & 0.333 \\
Cost \& Economic Viability & 0.134 & 0.205 & 0.000 & 0.000 & 0.250 \\
\hline
\multicolumn{6}{l}{\textsc{Panel D. Effect Shares}} \\
Revenue \& Financial Growth & 0.243 & 0.231 & 0.000 & 0.214 & 0.400 \\
Cost Reduction \& Efficiency & 0.114 & 0.174 & 0.000 & 0.000 & 0.200 \\
Market Expansion \& Adoption & 0.304 & 0.224 & 0.143 & 0.294 & 0.429 \\
Product \& Service Innovation & 0.235 & 0.212 & 0.000 & 0.214 & 0.333 \\
Operational Efficiency \& Automation & 0.104 & 0.159 & 0.000 & 0.000 & 0.176 \\
\hline
\multicolumn{6}{l}{\textsc{Panel E. Intensity Measures}} \\
Total span count, $N_{ikt}$ & 16.0 & 12.8 & 8.0 & 12.0 & 19.0 \\
Cause spans, $N^{\text{cause}}_{ikt}$ & 6.5 & 5.2 & 3.0 & 5.0 & 7.0 \\
Effect spans, $N^{\text{effect}}_{ikt}$ & 9.5 & 8.3 & 4.0 & 7.0 & 11.0 \\
Cause-to-effect ratio & 0.80 & 0.36 & 0.56 & 0.75 & 1.00 \\
\hline\hline
\end{tabular}%
}
\vspace{0.25em}
\end{table}

\clearpage
\begin{table}[!htbp]\centering
\caption{Macro Categories for Technology-Adoption Causes and Effects}
\label{tab:macro-categories}
\begin{flushleft}\footnotesize
\textit{Notes.} This table reports the distribution of extracted causal-language spans across macro categories.
The unit of observation is an upstream \emph{cause} span (Panel A) or a downstream \emph{effect} span (Panel B)
from an extracted causal triple after taxonomy assignment and macro-category mapping. Shares are computed as the
number of assigned spans in a macro category divided by the total number of spans on the same side (causes or
effects), pooled across all technologies, firms, and quarters.
\end{flushleft}
\tabfont
\begin{tabular*}{\textwidth}{@{\extracolsep{\fill}} L{0.33\textwidth} L{0.57\textwidth} P{0.08\textwidth}}
\hline\hline
Macro category & \multicolumn{1}{c}{Definition} & \multicolumn{1}{c}{Share (\%)} \\
\hline
\multicolumn{3}{l}{\textsc{Panel A. Causes of technology adoption}}\\
Technology Innovation and Advancement &
\parbox[t]{0.57\textwidth}{Advancements in technical performance, capabilities,\\
or underlying infrastructure that make adoption\\
feasible or attractive.} &
34.7 \\
Market Demand and Consumer Behavior &
\parbox[t]{0.57\textwidth}{Customer needs, preferences, competitive demand,\\
or market trends that pull firms\\
toward adoption.} &
28.9 \\
Strategic Partnerships and Investment &
\parbox[t]{0.57\textwidth}{Adoption driven by strategic choice,\\
complementary investments, alliances, acquisitions,\\
or ecosystem positioning.} &
17.6 \\
Cost and Economic Viability &
\parbox[t]{0.57\textwidth}{Adoption motivated by affordability,\\
expected cost--benefit tradeoffs, financing constraints,\\
or economic feasibility.} &
12.7 \\
Regulatory and Policy Drivers &
\parbox[t]{0.57\textwidth}{Adoption triggered by regulation,\\
compliance requirements, public policy incentives,\\
or government mandates.} &
6.1 \\
\hline
\multicolumn{3}{l}{\textsc{Panel B. Effects of technology adoption}}\\
Market Expansion and Adoption &
\parbox[t]{0.57\textwidth}{Growth in market share, customer base, geographic reach,\\
or penetration into new segments through\\
technology uptake.} &
29.3 \\
Product and Service Innovation &
\parbox[t]{0.57\textwidth}{Advancements in product features, design capabilities,\\
technical performance, or\\
new service offerings.} &
25.2 \\
Revenue and Financial Growth &
\parbox[t]{0.57\textwidth}{Direct increases in sales, revenue, profit margins,\\
or other financial performance metrics\\
from technology adoption.} &
22.2 \\
Operational Efficiency and Automation &
\parbox[t]{0.57\textwidth}{Streamlining business processes, workflow improvements,\\
productivity gains, and automation\\
of manual tasks.} &
12.3 \\
Cost Reduction and Efficiency &
\parbox[t]{0.57\textwidth}{Decreases in operational expenses, manufacturing costs,\\
or resource usage while improving\\
process efficiency.} &
10.9 \\
\hline\hline
\end{tabular*}
\vspace{0.25em}
\end{table}

\clearpage

\begin{table}[!htbp]\centering
\caption{Variance Decomposition of Macro Shares}
\label{tab:vd_macroshares}
\begin{flushleft}\footnotesize
\textit{Notes.} This table reports the incremental \(R^{2}\) shares from projecting the firm--technology--quarter macro-share outcomes on nested sets of fixed effects (Hassan et al. style). Columns in Panel A correspond to the five \emph{cause} macro categories; columns in Panel B correspond to the five \emph{effect} macro categories. Industry granularity is fixed at sic4. All components are computed on the same estimation sample as the most saturated model using \texttt{e(sample)}.
\end{flushleft}
\tabfont
\begin{tabular}{lccccc}
\hline\hline
\multicolumn{6}{l}{\textsc{Panel A. Cause}}\\ 
 & Technology Innovation and Advancement & Market Demand and Consumer Behavior & Regulatory and Policy Drivers & Strategic Partnerships and Investment & Cost and Economic Viability \\\\
 & (1) & (2) & (3) & (4) & (5) \\\\
\hline
Time FE & 1.89\% & 3.28\% & 1.08\% & 1.37\% & 1.93\% \\\\
Industry FE & 8.27\% & 6.46\% & 7.40\% & 8.71\% & 6.99\% \\\\
Industry $\times$ time FE & 12.23\% & 12.24\% & 12.85\% & 11.87\% & 12.67\% \\\\
Technology FE & 19.27\% & 11.07\% & 6.11\% & 7.79\% & 11.67\% \\\\
Technology $\times$ time FE & 8.03\% & 7.18\% & 7.06\% & 6.52\% & 5.95\% \\\\
``Firm-level'' & 50.31\% & 59.77\% & 65.49\% & 63.73\% & 60.80\% \\\\
\quad \parbox[t]{0.62\linewidth}{Permanent differences across firms\\within industries (Firm FE)} & 13.35\% & 14.50\% & 20.59\% & 17.08\% & 16.46\% \\\\
\quad \parbox[t]{0.62\linewidth}{Time-varying firm component\\(residual)} & 36.97\% & 45.27\% & 44.90\% & 46.65\% & 44.34\% \\\\
Number of industries & 94 & 94 & 94 & 94 & 94 \\\\
\hline
\multicolumn{6}{l}{\textsc{Panel B. Effect}}\\ 
 & Revenue and Financial Growth & Cost Reduction and Efficiency & Market Expansion and Adoption & Product and Service Innovation & Operational Efficiency and Automation \\\\
 & (1) & (2) & (3) & (4) & (5) \\\\
\hline
Time FE & 0.80\% & 0.86\% & 1.14\% & 0.59\% & 2.62\% \\\\
Industry FE & 4.42\% & 6.35\% & 5.18\% & 5.16\% & 6.35\% \\\\
Industry $\times$ time FE & 11.91\% & 12.10\% & 12.19\% & 12.27\% & 10.68\% \\\\
Technology FE & 5.53\% & 12.91\% & 6.56\% & 4.97\% & 15.78\% \\\\
Technology $\times$ time FE & 5.32\% & 4.69\% & 6.07\% & 5.63\% & 4.16\% \\\\
``Firm-level'' & 72.02\% & 63.10\% & 68.86\% & 71.39\% & 60.40\% \\\\
\quad \parbox[t]{0.62\linewidth}{Permanent differences across firms\\within industries (Firm FE)} & 20.77\% & 13.68\% & 14.83\% & 17.30\% & 14.06\% \\\\
\quad \parbox[t]{0.62\linewidth}{Time-varying firm component\\(residual)} & 51.25\% & 49.42\% & 54.03\% & 54.08\% & 46.34\% \\\\
Number of industries & 173 & 173 & 173 & 173 & 173 \\\\
\hline\hline
\end{tabular}

\vspace{0.25em}
\end{table}

\clearpage
\begin{table}[!htbp]\centering
\caption{Cross-Technology Belief Spillover}
\label{tab:spillover}
\begin{flushleft}\footnotesize
\textit{Notes.} This table reports estimates from stacked regressions of a firm's own cause (Panel~A) or effect (Panel~B) share for technology~$k$ on the firm's leave-one-out peer share from its other technologies. Each observation is a firm--technology--quarter--category, where categories are the five macro cause or effect groups defined in Section~II.D. The peer share for firm~$i$ in category~$c$ excludes all observations for the same technology~$k$. All fixed effects are interacted with category. Column~(I) includes technology$\times$category and quarter$\times$category fixed effects. Column~(II) replaces quarter with industry$\times$quarter FE. Column~(III) further adds technology$\times$quarter FE, absorbing technology-specific trends. Column~(IV) restricts to the first quarter in which a firm discusses each technology. Standard errors clustered at the firm level are in parentheses. $^{***}$~$p<0.01$, $^{**}$~$p<0.05$, $^{*}$~$p<0.10$.
\end{flushleft}
\resizebox{\textwidth}{!}{%
\begin{tabular}{lcccc}
\hline\hline
\multicolumn{5}{l}{\textsc{Panel A. Cause shares}}\\
 & \parbox[b]{2.2cm}{\centering Baseline} & \parbox[b]{2.2cm}{\centering Ind x Qtr} & \parbox[b]{2.2cm}{\centering Tech x Qtr} & \parbox[b]{2.2cm}{\centering First appear.} \\
 & (I) & (II) & (III) & (IV) \\
\hline
Peer share & 0.065*** & 0.040** & 0.042** & 0.049** \\
 & (0.018) & (0.020) & (0.017) & (0.020) \\[6pt]
Technology $\times$ category FE & Yes & Yes & No & Yes \\
Quarter $\times$ category FE & Yes & No & No & Yes \\
Ind. $\times$ quarter $\times$ category FE & No & Yes & Yes & No \\
Technology $\times$ quarter $\times$ category FE & No & No & Yes & No \\
Observations & 36,950 & 36,950 & 36,950 & 9,625 \\
$R^2$ & 0.003 & 0.001 & 0.001 & 0.002 \\
\hline
\multicolumn{5}{l}{\textsc{Panel B. Effect shares}}\\
 & (I) & (II) & (III) & (IV) \\
\hline
Peer share & 0.099*** & 0.100*** & 0.103*** & 0.087*** \\
 & (0.017) & (0.018) & (0.019) & (0.019) \\[6pt]
Observations & 36,950 & 36,950 & 36,950 & 9,625 \\
$R^2$ & 0.007 & 0.007 & 0.008 & 0.005 \\
\hline\hline
\end{tabular}%
}
\vspace{0.25em}
\end{table}
\clearpage
\clearpage
\input{Appendix/top_scoring_transcripts}

\clearpage
\begin{figure}[p]
\caption{Technology-Specific Cause Categories over Time}
\label{fig:app-pertech-cause}
\vspace{0.5em}
\centering
\includegraphics[width=\textwidth,page=1]{PerTech_category_top5_Cause.pdf}

\begin{flushleft}\footnotesize
\textit{Notes.} Each page shows a stacked area chart for one of the 29 technologies.
The vertical axis is the share of cause-side spans assigned to each of the top five technology-specific cause categories.
Shares are computed annually, pooled across all firms discussing the technology. The remaining 28 technologies follow on subsequent pages.
See Section~III.B for discussion.
\end{flushleft}
\end{figure}

\includepdf[pages=2-29,pagecommand={}]{Figures/PerTech_category_top5_Cause.pdf}

\clearpage
\begin{figure}[p]
\caption{Technology-Specific Effect Categories over Time}
\label{fig:app-pertech-effect}
\vspace{0.5em}
\centering
\includegraphics[width=\textwidth,page=1]{PerTech_category_top5_Effect.pdf}

\begin{flushleft}\footnotesize
\textit{Notes.} Each page shows a stacked area chart for one of the 29 technologies.
The vertical axis is the share of effect-side spans assigned to each of the top five technology-specific effect categories.
Shares are computed annually, pooled across all firms discussing the technology. The remaining 28 technologies follow on subsequent pages.
See Section~III.B for discussion.
\end{flushleft}
\end{figure}

\includepdf[pages=2-29,pagecommand={}]{Figures/PerTech_category_top5_Effect.pdf}




\end{document}
