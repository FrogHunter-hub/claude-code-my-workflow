\documentclass[11pt]{article}
\usepackage[utf8]{inputenc}
\usepackage[a4paper,margin=1in]{geometry}
\usepackage{enumitem}
\usepackage{titlesec}
\usepackage{hyperref}
\usepackage{lmodern}
\usepackage[T1]{fontenc}

% Tables and floats
\usepackage{booktabs}
\usepackage{array}
\usepackage{float}
\usepackage{placeins}
\usepackage{caption}
\usepackage{xcolor}

% Code block formatting
\usepackage{fvextra}
\fvset{
  fontsize=\small,
  commandchars=\\\{\},
  breaklines=true,
  breakanywhere=true,
  breaksymbol=\textcolor{gray}{\tiny\ensuremath{\hookrightarrow}},
  breaksymbolsep=2pt
}

\hypersetup{colorlinks=true, urlcolor=blue, linkcolor=blue}
\setlist[itemize]{leftmargin=1.2em}
\titleformat{\section}{\large\bfseries}{}{0em}{}

\title{Prompts for Harmonization and Aggregation Task}
\author{Xingxu Chai}
\date{2025.11.22}

\begin{document}
\maketitle

\section{System Prompt}
\begin{Verbatim}
You are a careful assistant. Always follow the user's prompt format exactly. 
Reply in English only.
\end{Verbatim}

\section{I.Canonicalize Cross-Tech categories}

% \subsection{For CAUSE side}
\begin{Verbatim}
<instruction>
**Goal**: Cluster the input items into a concise set of canonical labels for 
the given use case.
</instruction>

<use_case>
Create a conservative set of canonical labels to harmonize technology-adoption 
causes across technologies.
</use_case>

<requirements>
### Format
- Return the canonical groups as a **markdown table** with each row as a label, 
  with the following columns:
  - **canonical_id**: index starting from 1 in an incremental manner.
  - **canonical_name**: within 5 words.
  - **canonical_description**: within 30 words.
- Here is an example of your output:
```markdown
|canonical_id|canonical_name|canonical_description|
|-|-|-|
|1|...|...|
```
- Output table should be in English only.
- Normalize " & " to " and " in names.

### Quality
- No overlap or contradiction among the categories.
- Names are concise and specific; descriptions differentiate categories.
- Categories can accurately and consistently classify new data without ambiguity.
- No vague buckets such as Other, General, Unclear, Miscellaneous, Undefined.
- Be conservative: merge only when the names mean almost the same thing and 
  have the same scope.
- When the name or description has a technology-specific meaning, keep it 
  separate.
</requirements>

<data>
Format per line: technology||category_id||name||description
TECH_A||1||Category Name A||Description A
TECH_B||2||Category Name B||Description B
</data>

<questions>
Q1: Produce the canonical group table that meets all requirements.
</questions>

<output>
<canonical_groups>
|canonical_id|canonical_name|canonical_description|
|-|-|-|
</canonical_groups>
</output>
\end{Verbatim}


\section{II.Membership Mapping}

% \subsection{For CAUSE side}
\begin{Verbatim}
<instruction>
Goal: Review the reference table and produce a membership mapping for the 
given use case.
Decide primarily by NAME and DESCRIPTION; use examples for secondary validation.
</instruction>

<use_case>
Harmonize per-technology cause labels to canonical labels when the names mean 
almost the same thing and have the same scope; otherwise KEEP. Preserve 
technology-specific meaning.
</use_case>

<requirements>
### Format
- Output the mapping as a **markdown table** with each row as a source label, 
  with the following columns:
  - **technology**: technology name.
  - **category_id**: id from the per-technology table.
  - **original_name**: original label name.
  - **original_description**: original description.
  - **decision**: either HARMONIZE or KEEP.
  - **canonical_id**: id from the canonical table (blank if KEEP).
  - **new_name**: final label name (if KEEP, then it remains original_name; 
    if HARMONIZE, then it changes to canonical_name).
  - **new_description**: final description (if KEEP, then it remains 
    original_description; if HARMONIZE, then it changes to canonical_description).
  - **rationale**: \leq 20 words; brief justification of the chosen category.

Here is an example of your output:
```markdown
|technology|category_id|original_name|original_description|decision|canonical_id|new_name|new_description|rationale|
|-|-|-|-|-|-|-|-|-|
|TECH_A|12|...|...|HARMONIZE|1|...|...|...|
```
- Output table should be in English only.
- Normalize " & " to " and " in names.

### Quality
- **No overlap or contradiction** among the categories.
- **Names** are concise and specific; descriptions differentiate categories.
- **Categories** can accurately and consistently classify new data without 
  ambiguity.
- No vague buckets such as Other, General, Unclear, Miscellaneous, Undefined.
- When a label has a technology-specific meaning, default to KEEP unless the 
  chosen canonical has the same scope.
- Be conservative: if uncertain, select KEEP.
- Do not change per-technology ids or counts.
- Within the same technology and the same canonical label, keep at most one 
  HARMONIZE.
</requirements>

<reference_table>
|canonical_id|canonical_name|canonical_description|
|-|-|-|
|1|...|...|
</reference_table>

<data>
Format per line: technology||category_id||name||description||semicolon-separated examples
TECH_A||1||Name A||Desc A||example1; example2
TECH_B||2||Name B||Desc B||example3; example4
</data>

<questions>
Q1: Generate the membership mapping table that meets all requirements.
</questions>

<output>
|technology|category_id|original_name|original_description|decision|canonical_id|new_name|new_description|rationale|
|-|-|-|-|-|-|-|-|-|
</output>
\end{Verbatim}


\section{III.Macro Categories Design}

\subsection{For CAUSE side}
\begin{Verbatim}
<instruction>
Goal: Cluster the input items into a compact macro taxonomy for the given 
use case.
</instruction>

<use_case>
Design a small set of macro categories with clear economic meaning for 
technology-adoption causes.
</use_case>

<requirements>
### Format
Output clusters as a markdown table with each row as a category, with the 
following columns:
- **id**: category index starting from 1 in an incremental manner.
- **name**: category name should be within 5 words. It can be either verb phrase 
  or noun phrase, whichever is more appropriate.
- **description**: category description should be within 30 words.
- **pos_examples**: up to 3 phrases copied verbatim from the provided input data 
  that clearly belong to this category; semicolon-separated.

Here is an example of your output:
```markdown
|id|name|description|pos_examples|
|-|-|-|-|
|1|...|...|ex1; ex2|
```
Total number of categories must be \leq 10.
- Output table should be in English only.
- Normalize " & " to " and " in names.

### Quality
- **No overlap or contradiction** among the categories.
- **Names** are concise and specific; descriptions differentiate categories.
- **Categories** can accurately and consistently classify new data without 
  ambiguity.
- No vague buckets such as Other, General, Unclear, Miscellaneous, Undefined.
- Keep categories economically meaningful.
</requirements>

<data>
Format per line: technology||category_id||name||description||semicolon-separated examples
TECH_A||1||Name A||Desc A||ex1; ex2
TECH_B||2||Name B||Desc B||ex3; ex4
</data>

<questions>
Q1: Generate the macro taxonomy table that meets all requirements.
Q2: Briefly explain (\leq 80 words) how the boundaries are economically 
meaningful.
</questions>

<output>
<cluster_table>
|id|name|description|pos_examples|
|-|-|-|-|
</cluster_table>
<explanation></explanation>
</output>
\end{Verbatim}


\section{IV.Macro Mapping}

% \subsection{For CAUSE side}
\begin{Verbatim}
<instruction>
Use the reference taxonomy to assign each item to ONE best-fitting macro 
category id.
If none fits perfectly, choose the closest by the definitions.
Return ONLY the assignments table below.
</instruction>

<use_case>
Assign harmonized cause labels to one best-fitting macro category.
</use_case>

<requirements>
### Format
- Output the assignments as a **markdown table** with each row as an assignment, 
  with the following columns:
  - **technology**: technology name.
  - **category_id**: id from the harmonized per-technology table.
  - **category_name**: name from the harmonized per-technology table.
  - **macro_id**: id from the reference macro taxonomy.
  - **macro_name**: name from the reference macro taxonomy.
  - **rationale**: \leq 20 words; brief justification of the chosen macro category.

Here is an example of your output:
```markdown
|technology|category_id|category_name|macro_id|macro_name|rationale|
|-|-|-|-|-|-|
|TECH_A|7|...|3|...|...|
```
- Output table should be in English only.
- Normalize " & " to " and " in names.

### Quality
- Cover every input item exactly once.
- Choose the strictest valid macro category; keep boundaries consistent.
- Prioritize economic definitions.
</requirements>

<reference_table>
|id|name|description|pos_examples|
|-|-|-|-|
|1|...|...|ex1; ex2|
</reference_table>

<data>
Format per line: technology||category_id||name||description
TECH_A||1||Name A||Desc A
TECH_B||2||Name B||Desc B
</data>

<questions>
Q1: Return the assignments table that meets all requirements.
</questions>

<output>
|technology|category_id|category_name|macro_id|macro_name|rationale|
|-|-|-|-|-|-|
</output>
\end{Verbatim}



\end{document}