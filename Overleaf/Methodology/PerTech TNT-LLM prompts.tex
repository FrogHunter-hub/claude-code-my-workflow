\documentclass[11pt]{article}
\usepackage[utf8]{inputenc}
\usepackage[a4paper,margin=1in]{geometry}
\usepackage{enumitem}
\usepackage{titlesec}
\usepackage{hyperref}
\usepackage{lmodern}
\usepackage[T1]{fontenc}

% 表格与浮动体
\usepackage{booktabs}
\usepackage{array}
\usepackage{float}
\usepackage{placeins}
\usepackage{caption}
\usepackage{xcolor}

% 代码块排版
\usepackage{fvextra}
\fvset{
  fontsize=\small,
  commandchars=\\\{\},
  breaklines=true,
  breakanywhere=true,
  breaksymbol=\textcolor{gray}{\tiny\ensuremath{\hookrightarrow}},
  breaksymbolsep=2pt
}

\hypersetup{colorlinks=true, urlcolor=blue, linkcolor=blue}
\setlist[itemize]{leftmargin=1.2em}
\titleformat{\section}{\large\bfseries}{}{0em}{}

\title{Per-Technology TNT Prompt}
\author{}
\date{2025.11.19}

\begin{document}
\maketitle

\section{System Prompt (applied to all calls)}
\begin{Verbatim}
"You are a careful assistant. Always follow the user's prompt format exactly. Reply in English only."
\end{Verbatim}

% =========================================================
% I. Per-Technology TNT (Phase-I)
% =========================================================
\section{i.Per-Technology TNT (Phase-I) on 29 Techs}

\subsection{CAUSE(EFFECT) — Generation}
\begin{Verbatim}
<instruction>
**Goal**: Your goal is to cluster the input data into meaningful categories for the given use case.
</instruction>
<use_case>Classify short phrases describing technology-adoption causes (drivers/motives) in <<TECH>> extracted from earnings calls.</use_case>
<requirements>
### Format
- Output clusters as a **markdown table** with each row as a category, with the following columns:
  - **id**: category index starting from 1 in an incremental manner.
  - **name**: category name should be **within 5 words**. It can be either *verb phrase* or *noun phrase*, whichever is more appropriate.
  - **description**: category description should be **within 30 words**.
  - **inclusion_rules**: 2–4 short rules; semicolon-separated; criteria that qualify a phrase for the category.
  - **exclusion_rules**: 1–3 short rules; semicolon-separated; typical traps that should NOT be in the category.
  - **pos_examples**: up to 3 phrases **copied verbatim from the provided input data** that clearly belong to this category; semicolon-separated.
  - **near_misses**: up to 3 phrases **copied verbatim from the provided input data** that are borderline or often confused with this category but should go elsewhere; semicolon-separated.

Here is an example of your output:
```markdown
|id|name|description|inclusion_rules|exclusion_rules|pos_examples|near_misses|
|-|-|-|-|-|-|-|
|category id|category name|category description|rule A; rule B|trap A|example 1; example 2|near 1; near 2|
```
- Total number of categories should be no more than 15.
- Output table should be in English only.

### Quality
- **No overlap or contradiction** among the categories.
- **Name** is a concise and clear label for the category. Use only phrases that are specific to each category and avoid those that are common to all categories.
- **Description** differentiates one category from another.
- **Name** and **description** can **accurately** and **consistently** classify new data points **without ambiguity**.
- **Name** and **description** are *consistent with each other*.
- Output clusters match the data as closely as possible,without missing important categories or adding unnecessary ones.
- Output clusters serve the given use case well.
- Output clusters should be specific and meaningful. Do not invent categories that are not in the data.
</requirements>
<data>
|id|text|
|-|-|
|1|...|
</data>
<questions>
Q1: Generate the taxonomy table that meets all requirements.
Tips:
- The cluster table should be a flat list of mutually exclusive categories. Sort them based on their semantic relatedness.
- You can have fewer than 15 categories, but do not exceed the limit.
- Be specific about each category. Do not include vague categories such as "Other", "General", "Unclear", "Miscellaneous" or "Undefined".
- You can ignore low quality or ambiguous data points.
Q2: Why did you cluster the data the way you did? Explain your reasoning within 120 words.
</questions>
<output>
<cluster_table>
|id|name|description|inclusion_rules|exclusion_rules|pos_examples|near_misses|
|-|-|-|-|-|-|-|
</cluster_table>
<explanation></explanation>
</output>
\end{Verbatim}

\subsection{CAUSE(EFFECT) — Update}
\begin{Verbatim}
<instruction>
**Goal**: Your goal is to review the given reference table based on the input data for the specified use case, then update the reference table if needed.
- You will be given a reference cluster table, which is built on existing data. The reference table will be used to classify new data points.
- You will compare the input data with the reference table, output a rating score of the quality of the reference table, suggest potential edits, and update the reference table if needed.
- Reference cluster table: markdown table with columns: id, name, description, inclusion_rules, exclusion_rules, pos_examples, near_misses.
</instruction>
<use_case>Classify short phrases describing technology-adoption causes (drivers/motives) in <<TECH>> extracted from earnings calls.</use_case>
<requirements>
### Format
- Output clusters as a **markdown table** with each row as a category, with the following columns:
  - **id**: category index starting from 1 in an incremental manner.
  - **name**: category name should be **within 5 words**. It can be either *verb phrase* or *noun phrase*, whichever is more appropriate.
  - **description**: category description should be **within 30 words**.
  - **inclusion_rules**: 2–4 short rules; semicolon-separated; criteria that qualify a phrase for the category.
  - **exclusion_rules**: 1–3 short rules; semicolon-separated; typical traps that should NOT be in the category.
  - **pos_examples**: up to 3 phrases **copied verbatim from the provided input data** that clearly belong to this category; semicolon-separated.
  - **near_misses**: up to 3 phrases **copied verbatim from the provided input data** that are borderline or often confused with this category but should go elsewhere; semicolon-separated.
- Total number of categories should be no more than 15.
- Output table should be in English only.

### Quality
- **No overlap or contradiction** among the categories.
- Names are concise and specific; descriptions differentiate categories.
- Categories can accurately and consistently classify new data without ambiguity.
- No vague buckets such as Other, General, Unclear, Miscellaneous, Undefined.
</requirements>
<previous_errors>
(Optional: only present when validation errors exist.)
</previous_errors>
<reference_table>
|id|name|description|inclusion_rules|exclusion_rules|pos_examples|near_misses|
|-|-|-|-|-|-|-|
|1|...|...|...|...|...|...|
</reference_table>
<data>
|id|text|
|-|-|
|1|...|
</data>
<questions>
Q1. Review the given reference table and the input data and provide a rating score of the reference table. The rating score should be an integer between 0 and 100, higher rating score means better quality. You should consider the following factors when rating the reference cluster table:
- Intrinsic quality:
 - 1) if the cluster table meets the Requirements section, with clear and consistent category names and descriptions, and no overlap or contradiction among the categories;
 - 2) if the categories in the cluster table are relevant to the given use case;
 - 3) if the cluster table includes any vague categories such as "Other", "General", "Unclear", "Miscellaneous" or "Undefined".
- Extrinsic quality:
 - 1) if the cluster table can accurately and consistently classify the input data without ambiguity;
 - 2) if there are missing categories in the cluster table that appear in the input data;
 - 3) if there are unnecessary categories in the cluster table that do not appear in the input data.
Q2. Explain your rating score in Q1 within 120 words.
Q3. Based on your review, decide if you need to edit the reference table to improve its quality. If yes, suggest potential edits within 120 words. If no, please output "N/A".
Tips:
- You can edit the category name, description, or remove a category.
- You can also merge or add new categories if needed. Your edits should meet the Requirements section.
- You can edit inclusion_rules, exclusion_rules, pos_examples and near_misses if needed.
- The cluster table should be a **flat list** of **mutually exclusive** categories. Sort them based on their semantic relatedness.
- You can have *fewer than 15 categories*, but **do not exceed the limit**.
- Be **specific** about each category. **Do not include vague categories** such as "Other", "General", "Unclear", "Miscellaneous" or "Undefined".
- You can ignore low quality or ambiguous data points.
Q4. If you decide to edit the reference table, please provide your updated reference table. If you decide not to edit the reference table, please output the original reference table.
</questions>
<output>
<rating_score></rating_score>
<explanation></explanation>
<suggestions></suggestions>
<cluster_table>
|id|name|description|inclusion_rules|exclusion_rules|pos_examples|near_misses|
|-|-|-|-|-|-|-|
</cluster_table>
</output>
\end{Verbatim}

\subsection{CAUSE(EFFECT) — Final Review}
\begin{Verbatim}
<instruction>
**Goal**: Your goal is to review the given reference table based on the requirements and the specified use case, then update the reference table if needed.
- You will be given a reference cluster table, which is built on existing data. The reference table will be used to classify new data points.
- You will compare the reference table with the given requirements, output a rating score of the quality of the reference table, suggest potential edits, and update the reference table if needed.
- Keep the 7-column header exactly: id, name, description, inclusion_rules, exclusion_rules, pos_examples, near_misses.
</instruction>
<use_case>Classify short phrases describing technology-adoption causes (drivers/motives) in <<TECH>> extracted from earnings calls.</use_case>
<requirements>
### Format
- Output clusters as a **markdown table** with each row as a category, with the following columns:
  - **id**: category index starting from 1 in an incremental manner.
  - **name**: category name should be **within 5 words**. It can be either *verb phrase* or *noun phrase*, whichever is more appropriate.
  - **description**: category description should be **within 30 words**.
  - **inclusion_rules**: 2–4 short rules; semicolon-separated; criteria that qualify a phrase for the category.
  - **exclusion_rules**: 1–3 short rules; semicolon-separated; typical traps that should NOT be in the category.
  - **pos_examples**: up to 3 phrases **copied verbatim from the provided input data** that clearly belong to this category; semicolon-separated.
  - **near_misses**: up to 3 phrases **copied verbatim from the provided input data** that are borderline or often confused with this category but should go elsewhere; semicolon-separated.
- Total number of categories should be no more than 15.
- Output table should be in English only.

### Quality
- **No overlap or contradiction** among the categories.
- Names are concise and specific; descriptions differentiate categories.
- Categories can accurately and consistently classify new data without ambiguity.
- No vague buckets such as Other, General, Unclear, Miscellaneous, Undefined.
</requirements>
<previous_errors>
(Optional: only present when validation errors exist.)
</previous_errors>
<reference_table>
|id|name|description|inclusion_rules|exclusion_rules|pos_examples|near_misses|
|-|-|-|-|-|-|-|
|1|...|...|...|...|...|...|
</reference_table>
<questions>
Q1. Review the given reference table and provide a rating score. The rating score should be an integer between 0 and 100, higher rating score means better quality. You should consider the same factors as in the update prompt.
Q2. Explain your rating score in Q1 within 120 words.
Q3. Based on your review, decide if you need to edit the reference table to improve its quality. If yes, suggest potential edits within 120 words. If no, please output "N/A".
Tips:
- You can edit the category name, description, or remove a category.
- You can also merge or add new categories if needed. Your edits should meet the Requirements section.
- You can edit inclusion_rules, exclusion_rules, pos_examples and near_misses if needed.
- The cluster table should be a **flat list** of **mutually exclusive** categories. Sort them based on their semantic relatedness.
- You can have *fewer than 15 categories*, but **do not exceed the limit**.
- Be **specific** about each category. **Do not include vague categories** such as "Other", "General", "Unclear", "Miscellaneous" or "Undefined".
- You can ignore low quality or ambiguous data points.
Q4. If you decide to edit the reference table, please provide your updated reference table. If you decide not to edit the reference table, please output the original reference table.
</questions>
<output>
<rating_score></rating_score>
<explanation></explanation>
<suggestions></suggestions>
<cluster_table>
|id|name|description|inclusion_rules|exclusion_rules|pos_examples|near_misses|
|-|-|-|-|-|-|-|
</cluster_table>
</output>
\end{Verbatim}

\subsection{CAUSE(EFFECT) — A/B Evaluation}
\begin{Verbatim}
<instruction>
Pick which taxonomy (1 or 2) better fits the validation data and meets all requirements.
Return ONLY the number "1" or "2" in <choice>, followed by a brief explanation.
</instruction>
<use_case>Classify short phrases describing technology-adoption causes (drivers/motives) in <<TECH>> extracted from earnings calls.</use_case>
<requirements>
### Format
- Output clusters as a **markdown table** with each row as a category, with the following columns:
  - **id**: category index starting from 1 in an incremental manner.
  - **name**: category name should be **within 5 words**. It can be either *verb phrase* or *noun phrase*, whichever is more appropriate.
  - **description**: category description should be **within 30 words**.
  - **inclusion_rules**: 2–4 short rules; semicolon-separated; criteria that qualify a phrase for the category.
  - **exclusion_rules**: 1–3 short rules; semicolon-separated; typical traps that should NOT be in the category.
  - **pos_examples**: up to 3 phrases **copied verbatim from the provided input data** that clearly belong to this category; semicolon-separated.
  - **near_misses**: up to 3 phrases **copied verbatim from the provided input data** that are borderline or often confused with this category but should go elsewhere; semicolon-separated.
- Total number of categories should be no more than 15.
- Output table should be in English only.

### Quality
- **No overlap or contradiction** among the categories.
- Names are concise and specific; descriptions differentiate categories.
- Categories can accurately and consistently classify new data without ambiguity.
- No vague buckets such as Other, General, Unclear, Miscellaneous, Undefined.
</requirements>
<taxonomy_1>
|id|name|description|inclusion_rules|exclusion_rules|pos_examples|near_misses|
|-|-|-|-|-|-|-|
|1|...|...|...|...|...|...|
</taxonomy_1>
<taxonomy_2>
|id|name|description|inclusion_rules|exclusion_rules|pos_examples|near_misses|
|-|-|-|-|-|-|-|
|1|...|...|...|...|...|...|
</taxonomy_2>
<data>
|id|text|
|-|-|
|1|...|
</data>
<output>
<choice>1 or 2</choice>
<explanation>(<=80 words)</explanation>
</output>
\end{Verbatim}

\subsection{CAUSE(EFFECT) — Quick Assignment}
\begin{Verbatim}
<instruction>
Use the reference taxonomy to assign each item to ONE best-fitting category id.
If none fits, set the category id to "Undefined".
Return ONLY the assignments table below.
</instruction>
<reference_table>
|id|name|description|inclusion_rules|exclusion_rules|pos_examples|near_misses|
|-|-|-|-|-|-|-|
|1|...|...|...|...|...|...|
</reference_table>
<data>
|id|text|
|-|-|
|1|...|
</data>
<output>
<assignments>
|id|predicted_category_id|predicted_category_name|
|-|-|-|
</assignments>
</output>
\end{Verbatim}

% \subsection{EFFECT — Generation}
% \begin{Verbatim}
% <instruction>
% **Goal**: Your goal is to cluster the input data into meaningful categories for the given use case.
% </instruction>
% <use_case>Classify short phrases describing technology-adoption effects (outcomes/impacts) in <<TECH>> extracted from earnings calls.</use_case>
% <requirements>
% ### Format
% - Output clusters as a **markdown table** with each row as a category, with the following columns:
%   - **id**: category index starting from 1 in an incremental manner.
%   - **name**: category name should be **within 5 words**. It can be either *verb phrase* or *noun phrase*, whichever is more appropriate.
%   - **description**: category description should be **within 30 words**.
%   - **inclusion_rules**: 2–4 short rules; semicolon-separated; criteria that qualify a phrase for the category.
%   - **exclusion_rules**: 1–3 short rules; semicolon-separated; typical traps that should NOT be in the category.
%   - **pos_examples**: up to 3 phrases **copied verbatim from the provided input data** that clearly belong to this category; semicolon-separated.
%   - **near_misses**: up to 3 phrases **copied verbatim from the provided input data** that are borderline or often confused with this category but should go elsewhere; semicolon-separated.

% Here is an example of your output:
% ```markdown
% |id|name|description|inclusion_rules|exclusion_rules|pos_examples|near_misses|
% |-|-|-|-|-|-|-|
% |category id|category name|category description|rule A; rule B|trap A|example 1; example 2|near 1; near 2|
% ```
% - Total number of categories should be no more than 15.
% - Output table should be in English only.

% ### Quality
% - **No overlap or contradiction** among the categories.
% - **Name** is a concise and clear label for the category. Use only phrases that are specific to each category and avoid those that are common to all categories.
% - **Description** differentiates one category from another.
% - **Name** and **description** can **accurately** and **consistently** classify new data points **without ambiguity**.
% - **Name** and **description** are *consistent with each other*.
% - Output clusters match the data as closely as possible,without missing important categories or adding unnecessary ones.
% - Output clusters serve the given use case well.
% - Output clusters should be specific and meaningful. Do not invent categories that are not in the data.
% </requirements>
% <data>
% |id|text|
% |-|-|
% |1|...|
% </data>
% <questions>
% Q1: Generate the taxonomy table that meets all requirements.
% Tips:
% - The cluster table should be a flat list of mutually exclusive categories. Sort them based on their semantic relatedness.
% - You can have fewer than 15 categories, but do not exceed the limit.
% - Be specific about each category. Do not include vague categories such as "Other", "General", "Unclear", "Miscellaneous" or "Undefined".
% - You can ignore low quality or ambiguous data points.
% Q2: Why did you cluster the data the way you did? Explain your reasoning within 120 words.
% </questions>
% <output>
% <cluster_table>
% |id|name|description|inclusion_rules|exclusion_rules|pos_examples|near_misses|
% |-|-|-|-|-|-|-|
% </cluster_table>
% <explanation></explanation>
% </output>
% \end{Verbatim}

% \subsection{EFFECT — Update}
% \begin{Verbatim}
% <instruction>
% **Goal**: Your goal is to review the given reference table based on the input data for the specified use case, then update the reference table if needed.
% - You will be given a reference cluster table, which is built on existing data. The reference table will be used to classify new data points.
% - You will compare the input data with the reference table, output a rating score of the quality of the reference table, suggest potential edits, and update the reference table if needed.
% - Reference cluster table: markdown table with columns: id, name, description, inclusion_rules, exclusion_rules, pos_examples, near_misses.
% </instruction>
% <use_case>Classify short phrases describing technology-adoption effects (outcomes/impacts) in <<TECH>> extracted from earnings calls.</use_case>
% <requirements>
% ### Format
% - Output clusters as a **markdown table** with each row as a category, with the following columns:
%   - **id**: category index starting from 1 in an incremental manner.
%   - **name**: category name should be **within 5 words**. It can be either *verb phrase* or *noun phrase*, whichever is more appropriate.
%   - **description**: category description should be **within 30 words**.
%   - **inclusion_rules**: 2–4 short rules; semicolon-separated; criteria that qualify a phrase for the category.
%   - **exclusion_rules**: 1–3 short rules; semicolon-separated; typical traps that should NOT be in the category.
%   - **pos_examples**: up to 3 phrases **copied verbatim from the provided input data** that clearly belong to this category; semicolon-separated.
%   - **near_misses**: up to 3 phrases **copied verbatim from the provided input data** that are borderline or often confused with this category but should go elsewhere; semicolon-separated.
% - Total number of categories should be no more than 15.
% - Output table should be in English only.

% ### Quality
% - **No overlap or contradiction** among the categories.
% - Names are concise and specific; descriptions differentiate categories.
% - Categories can accurately and consistently classify new data without ambiguity.
% - No vague buckets such as Other, General, Unclear, Miscellaneous, Undefined.
% </requirements>
% <previous_errors>
% (Optional: only present when validation errors exist.)
% </previous_errors>
% <reference_table>
% |id|name|description|inclusion_rules|exclusion_rules|pos_examples|near_misses|
% |-|-|-|-|-|-|-|
% |1|...|...|...|...|...|...|
% </reference_table>
% <data>
% |id|text|
% |-|-|
% |1|...|
% </data>
% <questions>
% Q1. Review the given reference table and the input data and provide a rating score of the reference table. The rating score should be an integer between 0 and 100, higher rating score means better quality. You should consider the following factors when rating the reference cluster table:
% - Intrinsic quality:
%  - 1) if the cluster table meets the Requirements section, with clear and consistent category names and descriptions, and no overlap or contradiction among the categories;
%  - 2) if the categories in the cluster table are relevant to the given use case;
%  - 3) if the cluster table includes any vague categories such as "Other", "General", "Unclear", "Miscellaneous" or "Undefined".
% - Extrinsic quality:
%  - 1) if the cluster table can accurately and consistently classify the input data without ambiguity;
%  - 2) if there are missing categories in the cluster table that appear in the input data;
%  - 3) if there are unnecessary categories in the cluster table that do not appear in the input data.
% Q2. Explain your rating score in Q1 within 120 words.
% Q3. Based on your review, decide if you need to edit the reference table to improve its quality. If yes, suggest potential edits within 120 words. If no, please output "N/A".
% Tips:
% - You can edit the category name, description, or remove a category.
% - You can also merge or add new categories if needed. Your edits should meet the Requirements section.
% - You can edit inclusion_rules, exclusion_rules, pos_examples and near_misses if needed.
% - The cluster table should be a **flat list** of **mutually exclusive** categories. Sort them based on their semantic relatedness.
% - You can have *fewer than 15 categories*, but **do not exceed the limit**.
% - Be **specific** about each category. **Do not include vague categories** such as "Other", "General", "Unclear", "Miscellaneous" or "Undefined".
% - You can ignore low quality or ambiguous data points.
% Q4. If you decide to edit the reference table, please provide your updated reference table. If you decide not to edit the reference table, please output the original reference table.
% </questions>
% <output>
% <rating_score></rating_score>
% <explanation></explanation>
% <suggestions></suggestions>
% <cluster_table>
% |id|name|description|inclusion_rules|exclusion_rules|pos_examples|near_misses|
% |-|-|-|-|-|-|-|
% </cluster_table>
% </output>
% \end{Verbatim}

% \subsection{EFFECT — Final Review}
% \begin{Verbatim}
% <instruction>
% **Goal**: Your goal is to review the given reference table based on the requirements and the specified use case, then update the reference table if needed.
% - You will be given a reference cluster table, which is built on existing data. The reference table will be used to classify new data points.
% - You will compare the the reference table with the given requirements, output a rating score of the quality of the reference table, suggest potential edits, and update the reference table if needed.
% - Keep the 7-column header exactly: id, name, description, inclusion_rules, exclusion_rules, pos_examples, near_misses.
% </instruction>
% <use_case>Classify short phrases describing technology-adoption effects (outcomes/impacts) in <<TECH>> extracted from earnings calls.</use_case>
% <requirements>
% ### Format
% - Output clusters as a **markdown table** with each row as a category, with the following columns:
%   - **id**: category index starting from 1 in an incremental manner.
%   - **name**: category name should be **within 5 words**. It can be either *verb phrase* or *noun phrase*, whichever is more appropriate.
%   - **description**: category description should be **within 30 words**.
%   - **inclusion_rules**: 2–4 short rules; semicolon-separated; criteria that qualify a phrase for the category.
%   - **exclusion_rules**: 1–3 short rules; semicolon-separated; typical traps that should NOT be in the category.
%   - **pos_examples**: up to 3 phrases **copied verbatim from the provided input data** that clearly belong to this category; semicolon-separated.
%   - **near_misses**: up to 3 phrases **copied verbatim from the provided input data** that are borderline or often confused with this category but should go elsewhere; semicolon-separated.
% - Total number of categories should be no more than 15.
% - Output table should be in English only.

% ### Quality
% - **No overlap or contradiction** among the categories.
% - Names are concise and specific; descriptions differentiate categories.
% - Categories can accurately and consistently classify new data without ambiguity.
% - No vague buckets such as Other, General, Unclear, Miscellaneous, Undefined.
% </requirements>
% <previous_errors>
% (Optional: only present when validation errors exist.)
% </previous_errors>
% <reference_table>
% |id|name|description|inclusion_rules|exclusion_rules|pos_examples|near_misses|
% |-|-|-|-|-|-|-|
% |1|...|...|...|...|...|...|
% </reference_table>
% <questions>
% Q1. Review the given reference table and provide a rating score. The rating score should be an integer between 0 and 100, higher rating score means better quality. You should consider the same factors as in the update prompt.
% Q2. Explain your rating score in Q1 within 120 words.
% Q3. Based on your review, decide if you need to edit the reference table to improve its quality. If yes, suggest potential edits within 120 words. If no, please output "N/A".
% Tips:
% - You can edit the category name, description, or remove a category.
% - You can also merge or add new categories if needed. Your edits should meet the Requirements section.
% - You can edit inclusion_rules, exclusion_rules, pos_examples and near_misses if needed.
% - The cluster table should be a **flat list** of **mutually exclusive** categories. Sort them based on their semantic relatedness.
% - You can have *fewer than 15 categories*, but **do not exceed the limit**.
% - Be **specific** about each category. **Do not include vague categories** such as "Other", "General", "Unclear", "Miscellaneous" or "Undefined".
% - You can ignore low quality or ambiguous data points.
% Q4. If you decide to edit the reference table, please provide your updated reference table. If you decide not to edit the reference table, please output the original reference table.
% </questions>
% <output>
% <rating_score></rating_score>
% <explanation></explanation>
% <suggestions></suggestions>
% <cluster_table>
% |id|name|description|inclusion_rules|exclusion_rules|pos_examples|near_misses|
% |-|-|-|-|-|-|-|
% </cluster_table>
% </output>
% \end{Verbatim}

% \subsection{EFFECT — A/B Evaluation}
% \begin{Verbatim}
% <instruction>
% Pick which taxonomy (1 or 2) better fits the validation data and meets all requirements.
% Return ONLY the number "1" or "2" in <choice>, followed by a brief explanation.
% </instruction>
% <use_case>Classify short phrases describing technology-adoption effects (outcomes/impacts) in <<TECH>> extracted from earnings calls.</use_case>
% <requirements>
% ### Format
% - Output clusters as a **markdown table** with each row as a category, with the following columns:
%   - **id**: category index starting from 1 in an incremental manner.
%   - **name**: category name should be **within 5 words**. It can be either *verb phrase* or *noun phrase*, whichever is more appropriate.
%   - **description**: category description should be **within 30 words**.
%   - **inclusion_rules**: 2–4 short rules; semicolon-separated; criteria that qualify a phrase for the category.
%   - **exclusion_rules**: 1–3 short rules; semicolon-separated; typical traps that should NOT be in the category.
%   - **pos_examples**: up to 3 phrases **copied verbatim from the provided input data** that clearly belong to this category; semicolon-separated.
%   - **near_misses**: up to 3 phrases **copied verbatim from the provided input data** that are borderline or often confused with this category but should go elsewhere; semicolon-separated.
% - Total number of categories should be no more than 15.
% - Output table should be in English only.

% ### Quality
% - **No overlap or contradiction** among the categories.
% - Names are concise and specific; descriptions differentiate categories.
% - Categories can accurately and consistently classify new data without ambiguity.
% - No vague buckets such as Other, General, Unclear, Miscellaneous, Undefined.
% </requirements>
% <taxonomy_1>
% |id|name|description|inclusion_rules|exclusion_rules|pos_examples|near_misses|
% |-|-|-|-|-|-|-|
% |1|...|...|...|...|...|...|
% </taxonomy_1>
% <taxonomy_2>
% |id|name|description|inclusion_rules|exclusion_rules|pos_examples|near_misses|
% |-|-|-|-|-|-|-|
% |1|...|...|...|...|...|...|
% </taxonomy_2>
% <data>
% |id|text|
% |-|-|
% |1|...|
% </data>
% <output>
% <choice>1 or 2</choice>
% <explanation>(<=80 words)</explanation>
% </output>
% \end{Verbatim}

% \subsection{EFFECT — Quick Assignment}
% \begin{Verbatim}
% <instruction>
% Use the reference taxonomy to assign each item to ONE best-fitting category id.
% If none fits, set the category id to "Undefined".
% Return ONLY the assignments table below.
% </instruction>
% <reference_table>
% |id|name|description|inclusion_rules|exclusion_rules|pos_examples|near_misses|
% |-|-|-|-|-|-|-|
% |1|...|...|...|...|...|...|
% </reference_table>
% <data>
% |id|text|
% |-|-|
% |1|...|
% </data>
% <output>
% <assignments>
% |id|predicted_category_id|predicted_category_name|
% |-|-|-|
% </assignments>
% </output>
% \end{Verbatim}


\end{document}
