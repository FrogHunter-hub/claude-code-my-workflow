\documentclass[11pt]{article}
\usepackage[a4paper,margin=1in]{geometry}
\usepackage{booktabs}
\usepackage{longtable}
\usepackage{array}
\usepackage{multirow}
\usepackage{xcolor}
\usepackage{hyperref}
\usepackage{graphicx}
\usepackage{caption}
\usepackage{amsmath, amssymb}

\hypersetup{
  colorlinks=true,
  linkcolor=blue!60!black,
  urlcolor=blue!60!black,
  citecolor=blue!60!black
}

\title{Topic Modeling: A Comparative Report on LDA, NMF, BERTopic, and TNT-LLM}
\author{Xingxu Chai}
\date{2025.11.11}

\begin{document}
\maketitle

% \begin{abstract}
% We compare four topic modeling / labeling approaches for \emph{cross-technology causal classification} across 29 distinct technologies: \textbf{LDA}, \textbf{NMF}, \textbf{BERTopic}, and \textbf{TNT-LLM}. A key challenge is \textbf{technology-domain leakage}: models drift toward domain-driven clusters (e.g., \emph{smartphone}, \emph{video}, \emph{AI}, \emph{cloud}) rather than causal categories. To mitigate this (for all methods except TNT-LLM), we remove a curated tech-keyword list and also remove tokens that co-occur with the list and have cosine similarity $\ge 0.75$. Overall, \textbf{TNT-LLM} remains the strongest in readability and cross-domain alignment; \textbf{NMF} improves interpretability over LDA but shows \textbf{unbalanced cluster sizes}; \textbf{BERTopic} has strong semantic grouping yet more easily picks up domain features if not constrained.
% \end{abstract}

\section{Overall}
\textbf{Problem}: Because we want to discover \textbf{cross-technology} \emph{CAUSE} drivers and \emph{EFFECT} outcomes from short texts spanning 29 technologies. The common self-referential phrases (e.g., “release of video content promotes video”) often contain domain cues; If we use raw materials, models will cluster by \emph{technology domain} causes/effects instead of \emph{cross-technology} causes/effects.

\paragraph{Preprocess}
\begin{itemize}
  \item \textbf{Deduplicates}: Keep unique phrases for causes and effects (for all methods)
  \item \textbf{Keyword list removal}: delete the words apeared in the keyword list.(except for tnt-llm)
  \item \textbf{Similarity filtering}: drop tokens that highly co-occur with the list and have cosine similarity $\ge 0.75$.(except for tnt-llm)
\end{itemize}

\section{Results and Analysis}
\subsection{High-Level Takeaways}
\begin{itemize}
  \item \textbf{LDA}: Weakest. This method seems don't fit short-text tasks. It generates fuzzy topic boundaries and poor causal interpretability.
  \item \textbf{NMF}: Interpretability is better. But very \textbf{imbalanced clusters} (e.g., a large generic topic accounts for over 20\% among 20 topics ).
  \item \textbf{BERTopic}: Strong semantics. However, \emph{more susceptible to domain cues} even though after filtering.
  \item \textbf{TNT-LLM}: \textbf{Overall is best}. We can add 'cross-technology' requirement in prompt. An LLM proposes the unified label space and short definitions; a light classifier (e.g., Logistic/MLP) can then assign labels to the full corpus. Best readability and alignment across domains.

\end{itemize}

\section{Suggestions}
TNT-LLM. We can add 'cross-technology' requirement in prompt and generate the labels, labeling on sample.

\section{Examples}

\subsection{LDA: Examples and Issues}
LDA top-terms are broad and overlapping; category boundaries are unclear (excerpt):
\begin{quote}\small
\textbf{0} new, development, product, innovation, sales, ecosystem, across, revenue, operations, improved, time, unique, teams, legacy, chinese \\
\textbf{1} adoption, transition, increasing, increase, features, training, implementation, offering, million, enhanced, operating, users, demands, desire, robust \\
\textbf{2} business, team, optimization, initiatives, multi, related, like, automation, generation, workloads, analytics, moving, ip, premium, policy \\
\textbf{3} investment, strategic, industry, marketing, efforts, partners, existing, significant, continued, scaling, talent, regulations, advertising, expanded, capex \\
\textbf{4} high, cost, focus, performance, needs, prices, regulatory, tools, brand, engineering, projects, trend, events, material, base \\
\textbf{...}
\end{quote}

\subsection{NMF: Better Interpretability, But Unbalanced Clusters}
\begin{table}[h]
\centering
\caption{NMF ($K=20$), CAUSE side (excerpt)}
\begin{tabular}{@{}r r r p{8.5cm}@{}}
\toprule
\textbf{topic} & \textbf{num\_docs} & \textbf{doc\_share} & \textbf{top\_terms (truncated)} \\
\midrule
17 & 40416 & 0.2047 & increased, cost, requirements, increased demand, supply, increased investment, supply chain, capacity, reduction, sales \\
13 & 33977 & 0.1721 & new, product, technologies, launch, new products, applications, features, product launches \\
12 & 14942 & 0.0757 & market, market demand, shift, expansion, market growth, share, china, market adoption \\
16 & 13423 & 0.0680 & platform, integration, capabilities, platform development, software platform, deployment \\
7  & 10659 & 0.0540 & data, data center, analytics, data availability, infrastructure \\
\bottomrule
\end{tabular}
\end{table}

Interpretability is improved versus LDA, pointing to business/operational themes. However, Topic 17’s share is high ($\sim 20\%$), suggesting a need for \emph{regularization or post-hoc split} (e.g., hierarchical NMF, penalties, or reclustering within that topic).

\subsection{BERTopic: Strong Semantics, but more Domain-Prone. And also exist note MECE categories}
\begin{longtable}{@{}r r r p{8.6cm} p{5.2cm}@{}}
\caption{BERTopic (KMeans = 20), CAUSE side with LLM labels (all topics; top-terms truncated)}\label{tab:bertopic-full}\\
\toprule
\textbf{topic} & \textbf{num\_docs} & \textbf{doc\_share} & \textbf{top\_terms (truncated)} & \textbf{LLM label (short)} \\
\midrule
\endfirsthead
\toprule
\textbf{topic} & \textbf{num\_docs} & \textbf{doc\_share} & \textbf{top\_terms (truncated)} & \textbf{LLM label (short)} \\
\midrule
\endhead
0  & 21502 & 0.1088 & supply constraints, supply chain, production capacity, inventory, delivery, delays, cost reduction, issues, \dots & Supply Chain Efficiency Challenges \\
1  & 21345 & 0.1080 & customer demand, growing demand, customer needs, demand services, customer base, customer adoption, \dots & Growing Customer Demand \\
2  & 18111 & 0.0917 & vertical integration, integration, platform development, implementation, design wins, scalability, \dots & Product Development Integration \\
3  & 17702 & 0.0896 & investment, technology investments, innovation, capital investment, strategic investments, development, \dots & Strategic Investment in Innovation \\
4  & 13233 & 0.0670 & energy prices, oil prices, energy, gas prices, renewables, electricity, emissions, \dots & Energy Price Dynamics \\
5  & 13313 & 0.0674 & penetration, data traffic, bandwidth, coverage, low latency, connected devices, 802, \dots & Growing Connectivity Demand \\
6  & 10827 & 0.0548 & marketing efforts, marketing campaigns, strategy, spend, advertising, promotion, initiatives, \dots & Marketing Investment Growth \\
7  & 9296  & 0.0470 & partnership, partnerships, collaboration, strategic partnership, partnering, partners, \dots & Strategic Partnership Development \\
8  & 8953  & 0.0453 & regulatory requirements, compliance, approvals, regulations, standards, certifications, \dots & Regulatory Compliance Drivers \\
9  & 9144  & 0.0463 & migrating, migration, customer migration, on-premise, deployment, transformation, \dots & Customer Migration Strategies \\
10 & 8609  & 0.0436 & security requirements, security concerns, privacy, data protection, threats, sovereignty, \dots & Data Security Compliance \\
11 & 7482  & 0.0379 & computing power, low-power, power consumption, hardware, semiconductors, processors, \dots & Supply and Demand Dynamics \\
12 & 6858  & 0.0347 & ip portfolio, intellectual property, patent portfolio, licensing, ownership, proprietary, \dots & Intellectual Property Dynamics \\
13 & 7247  & 0.0367 & optimization, algorithms, ranking, distribution, recommendation, content management, \dots & Content Strategy Optimization \\
14 & 6275  & 0.0318 & competition, competitive environment, competitive pressure, pricing, competitive advantage, \dots & Intense Competitive Pressure \\
15 & 5308  & 0.0269 & lidar technology, lasers, optical, scanning, detection, phosphorescent materials, \dots & Advanced Detection and Visibility \\
16 & 3998  & 0.0202 & subsidies china, tariffs, government subsidies, china market, manufacturers, \dots & Government Subsidy Impact \\
17 & 3864  & 0.0196 & acquisition, acquisitions, acquiring, company acquisition, acquisition strategy, \dots & Strategic Business Acquisitions \\
18 & 2769  & 0.0140 & generative, ai-powered, ai-driven, machine intelligence, LLM, algorithms, \dots & Emerging Generative Capabilities \\
19 & 1757  & 0.0089 & covid pandemic, pandemic, lockdown, crisis, restrictions, \dots & Pandemic Impact and Response \\
\bottomrule
\end{longtable}

\subsection{TNT-LLM: Seems best fit overall}
\begin{longtable}{@{}r p{5.2cm} p{9.6cm}@{}}
\caption{TNT-LLM label space and concise definitions (all rows)}\label{tab:tnt-llm-full}\\
\toprule
\textbf{index} & \textbf{label\_name} & \textbf{definition ($\le$ 30 words)} \\
\midrule
\endfirsthead
\toprule
\textbf{index} & \textbf{label\_name} & \textbf{definition ($\le$ 30 words)} \\
\midrule
\endhead
1  & Acquisition of companies          & Purchase or integration of other firms to gain technology, market share, or capabilities. \\
2  & Partnership or alliance            & Collaboration with external entities to enhance offerings or market access. \\
3  & Market demand shift                & Changes in consumer or business preferences driving adoption of new technologies. \\
4  & Regulatory or policy change        & Government actions or standards creating needs for new technology solutions. \\
5  & Technological advancement          & Breakthroughs improving performance, efficiency, or feasibility of technologies. \\
6  & Operational efficiency need        & Desire to reduce costs or improve processes through technology adoption. \\
7  & Product launch or phase            & Introduction of new products requiring supporting technologies or capabilities. \\
8  & Supply chain factors               & Disruptions or constraints necessitating resilient technology approaches. \\
9  & Competitive differentiation        & Efforts to stand out via innovation or unique capabilities in the market. \\
10 & Customer-driven requirement        & Direct client demands for specific features or integration needs. \\
11 & Strategic business transformation  & Shifts in business models enabled by new technology adoption. \\
12 & Data and analytics capability      & Investments in data platforms or analysis to drive decisions and performance. \\
13 & Infrastructure scaling             & Expansion of IT or production infrastructure to handle growth. \\
14 & Security and compliance need       & Requirements for security, privacy, or regulatory compliance driving upgrades. \\
15 & Energy and sustainability focus    & Environmental goals or energy factors prompting adoption of eco-friendly tech. \\
16 & User behavior change               & Shifts in how users consume services, like mobile or streaming usage. \\
17 & Expertise and resource availability& In-house skills or dedicated teams facilitating technology implementation. \\
18 & Investment and funding initiative  & Capital investments, funding, or financial strategies enabling technology projects. \\
19 & Marketing and brand initiative     & Campaigns or branding efforts requiring supporting technology platforms. \\
20 & Undefined                          & Catch-all for outliers not fitting other categories. \\
\bottomrule
\end{longtable}

\end{document}
